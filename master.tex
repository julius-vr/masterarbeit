\documentclass[11pt]{book}
\usepackage[english]{babel}
\usepackage[all]{xy}
\usepackage{hyperref}
\usepackage{amsthm}
\usepackage{amsmath}
\usepackage{amssymb}

\newtheorem{prop}{Proposition}
\newtheorem{lemma}{Lemma}
\newtheorem{theorem}{Theorem}
\newtheorem{definition}{Definition}
\newtheorem{remark}{Remark}
\newtheorem{example}{Example}



\begin{document}

\tableofcontents

\chapter{Some PL topology}
\label{sec:Einleitung}

In this chapter we will collect some basic facts from the piecewise-linear category. Most of the proofs are omitted and can be found in standard references like \cite{pltopo}, \cite{hatcher}. If the reader is already familiar with basics from PL topology, the chapter may be skipped without loss of continuity.

\begin{definition}
Let $v_0,...,v_k \in \mathbb{R}^n$ be points in some affine space such that $\{ v_1-v_0,...,v_k-v_0 \}$ is a set of linearly independent vectors. We call

\begin{equation*}
[v_0,...,v_k] := \Biggl \{  \lambda_0v_0+...+ \lambda_kv_k \Bigg |  \sum_{i=0}^k \lambda_i =1 \text{ and } \lambda_i \geq 0 \text{ for all i} \Biggr \}
\end{equation*}
the simplex spanned by $\{v_0,...,v_k \}$. Its dimension is $k$ and we call it a \textbf{$k$-simplex} for short. The points that span a simplex are called vertices. For a simplex $\sigma$ we say that $\tau$ is a \textbf{face} of $\sigma$ if $\tau$ is a simplex spanned by a nonempty subset of the vertices of $\sigma$ and we abbreviate this by writing $\tau < \sigma$.
\end{definition}

\begin{definition}
A \textbf{simplicial complex} $K$ is a set of simplices that satisfies the following conditions:

\begin{itemize}
\item Every face of a simplex in $K$ is also contained in $K$.
\item The intersection of any two simplices $\sigma, \tau \in K$ is either empty or a face of both $\sigma$ and $\tau$.
\end{itemize}

We define the \textbf{plyhedron} of $K$ by $|K|:= \bigcup \{ \sigma | \sigma \in K \}$. The \textbf{p-skeleton} of $K$ is given by $K^{(p)}= \{ \sigma \in  K | \text{dim}(\sigma) \leq p \}$. A \textbf{subcomplex} of $K$ is a subset $L \subset K$ such that $L$ itself is a simplicial complex. We denote the set of vertices of $K$ by $V(K)$. The \textbf{dimension} of $K$ is the largest dimension of any simplex contained in $K$. If no such maximum exists, we say that $K$ is of dimension $\infty$. $K$ is said to be of \textbf{pure dimension} $n$ if every simplex of $K$ is a face of some $n$-simplex in $K$.
\end{definition}


\begin{definition}
Let $K$ be a simplicial complex of pure dimension $n$. The \textbf{boundary} $Bd(K)$ of $K$ is defined to be the (possibly empty) subcomplex of $K$ of pure dimension $n-1$, whose $(n-1)$-simplices are those $(n-1)$-simplices of $K$ which are incident to precisely one $n$-simplex in $K$.
\end{definition}

\begin{definition}
For a simplicial complex $K$ and a simplex $\sigma \in K$, we call
\begin{equation*}
St(\sigma,K)= \{ \rho \in K | \rho < \tau, \sigma < \tau \text{ for some } \tau \in K \}  
\end{equation*}
the \textbf{star} of $\sigma$ in $K$. The \textbf{link} of $\sigma$ in $K$ is given by
\begin{equation*}
Lk(\sigma,K) = \{ \rho \in St(\sigma,K) | \sigma \cap \rho = \emptyset \}.
\end{equation*}
\end{definition}

\begin{definition}
Suppose that there are two simplicial complexes $K$ and $K'$ such that $|K| = |K'|$. If every simplex of $K'$ is contained in some simplex of $K$, we say that $K'$ is a \textbf{subdivision} of $K$ and write $K' \lhd K$.
\end{definition}


\begin{example}
Given a simplicial complex $K$ there is always an inductive process that produces a subdivision of $K$. Assume that $K^{(p-1)}$ has already been subdivided and let $\sigma=[v_0,...,v_p]$ be a $p$-simplex in $K$. The point $\hat{\sigma}=\frac{1}{p+1} \sum_{i=0}^p v_i$ lies in the interior of $\sigma$ and is called its \textbf{barycenter}. The \textbf{barycentric subdivision} of $\sigma$ is the decomposition of $\sigma$ into the $p$-simplices $[\hat{\sigma},w_0,...,w_{p-1}]$ where, inductively, $[w_o,...,w_{p-1}]$ is a $(p-1)$-simplex in the barycentric subdivision of a face $[v_0,...,\overline{v_i},...,v_p]$. (In this notation, the vertex $v_i$ is omitted.) Continuing this procedure for every $p$-simplex $\sigma$ leads to a decomposition of all simplices in $K^{(p)}$. The induction starts at $p=0$ when the barycentric subdivision of a $0$-simplex $[v_0]$ is just $[v_0]$ itself. It is guaranteed that this process delivers a subdivision $K^1 \lhd K$, called the \textbf{first barycentric subdivision} of $K$. For details, see \cite{hatcher}. More generally, the $r$-th barycentric subdivision is inductively given by $K^r=(K^{r-1})^1$.
\end{example}


\begin{definition}
A topological space $X$ is said to be $\mathbf{triangulable}$ if there exists a simplicial complex $T$ and a homeomorphism $\phi : |T| \to X$. The triple $(T,X, \phi)$ is called a \textbf{triangulation} of $X$. In this situation we will simply say that $T$ is a triangulation of $X$, by abuse of notation.
\end{definition}


\begin{definition}
A \textbf{PL space} is a pair $(X,\mathcal{T})$ consisting of a topological space $X$ and a class $\mathcal{T}$ of locally finite triangulations of $X$ which satisfies the following conditions:
\begin{itemize}
\item If $T \in \mathcal{T}$ then $T' \in \mathcal{T}$ for any subdivision $T' \lhd T$.
\item If $T,T' \in \mathcal{T}$ then there exists $T'' \in \mathcal{T}$ such that both $T'' \lhd T$ and $T'' \lhd T'$.
\end{itemize}
We will simply write $X$ for a PL space $(X,\mathcal{T})$ if there is no danger of confusion. A \textbf{closed PL subspace} of $X$ is a subcomplex of a suitable triangulation of $X$.
\end{definition}

\begin{definition}
Given simplicial complexes $K$ and $L$ we call a map $f: |K| \to |L|$ \textbf{simplicial} if $f$ maps each simplex of $K$ linearly onto some simplex of $L$. A map $g: |K| \to |L|$ between PL spaces  is said to be a \textbf{PL map} if there exist subdivisions $K' \lhd K$ and $L' \lhd L$ such that $g: K' \to L' $ is simplicial.
\end{definition}
Note that a simplicial map $f: K \to L$ is given by linear extension of a (set-theoretic) function $V(K) \to V(L)$.

\begin{definition}
A \textbf{simplicial isomorphism} between two simplicial complexes $K$ and $L$ is given by a bijection $f : V(K) \to V(L)$ such that $[v_0,...,v_k]$ is a $k$-simplex in $K$ if and only if $[f(v_0),...,f(v_k)]$ is a $k$-simplex in $L$. In particular, extending $f$ linearly yields a homeomorphism between the underlying polyhedra $|K|$ and $|L|$. A map $g: |K| \to |L|$ between PL spaces is a \textbf{PL isomorphism} if there exist subdivisions $K' \lhd K$ and $L' \lhd L$ such that $g: |K'| \to |L'|$ is a simplicial isomorphism.
\end{definition}

\begin{theorem}(Simplicial Approximation Theorem).
Let $f: |K| \to |L|$ be a map between polyhedra. Then there exist subdivisions $K' \lhd K$ and $L' \lhd L$ and a simplicial map $g: |K'| \to |L'|$ which is $\epsilon$-homotopic to $f$, i.e. if $\epsilon: |L| \to (0,\infty)$ is a map, then there is a map $H: |K| \times [0,1] \to |L|$ with $H(|K| \times \{ 0\})=f$, $H(|K| \times \{ 1\})=g$ and diam$(H(\{ x \} \times [0,1])) < \epsilon(f(x))$.
\end{theorem}

\begin{proof}
See \cite{lecturenotes}, e.g.
\end{proof}

DIMENSION \\
PL MANIFOLDS

\begin{definition}
A \textbf{0-dimensional PL stratified pseudomanifold} is a countable set of points with the discrete topology. An \textbf{n-dimensional PL stratified pseudomanifold} $X$ is a PL space together with a filtration of closed PL subspaces
\begin{equation*}
X=X_n \supset X_{n-1} = X_{n-2} \supset ... \supset X_0 \supset X_{-1} = \emptyset
\end{equation*}
such that the following conditions are satisfied:
\begin{itemize}
\item Every $X_{n-k} - X_{n-k-1}$ is a (possibly empty) PL manifold of dimension $n-k$.
\item $X-X_{n-2}$ is dense in $X$.
\item \textbf{Local normal triviality:} For every point $x \in X_{n-k} - X_{n-k-1}$ there exists an open neighborhood $U$ of $x$  in $X$ and a compact PL stratified pseudomanifold $L$ of dimension $k-1$ with filtration
\begin{equation*}
L = L_{k-1} \supset L_{k-3} \supset ... \supset L_0 \supset L_{-1}= \emptyset
\end{equation*}
and a PL isomorphism
\begin{equation*}
\phi : U \to \mathbb{R}^{n-k} \times c^{\circ} L
\end{equation*}
(where $c^{\circ}$ denotes the open cone) which restricts to PL isomorphism $\phi_| : U \cap X_{n-l} \to \mathbb{R}^{n-k} \times c^{\circ} L_{k-l-1}$. We say that $\phi$ is \textbf{stratum-preserving}.
\end{itemize}
\end{definition}





\chapter{A bordism approach to homology theories}
The purpose of this chapter is to give a geometric treatment of homology theories, as described by S. Buoncristiano, C. Rourke and B. Sanderson in \cite{BRS}. First we use local link properties to determine a class of polyhedra. Then, in an analagous manner to ordinary bordism theory, we define groups of bordism classes of maps, whose domains lie in the polyhedral class defined before. We observe that interesting examples of generalized homology theories can be interpreted in this way, including ordinary homology, ($\mathbb{Z}/2$)-homology and PL bordism theory. \newline
For the rest of this chapter we make the convention that every polyhedron is PL and of pure dimension.

\begin{definition}
Suppose we are given a class $\mathcal{L}_n$ of $(n-1)$-polyhedra which is closed under PL isomorphism. Then a \textbf{closed $\mathcal{L}_n$-manifold} is a polyhedron $M$ such that the link of each vertex of $M$ lies in $\mathcal{L}_n$. 
\end{definition}

\begin{definition}\label{theory}
A \textbf{theory with singularities} $\mathcal{L}$ consists of a class $\mathcal{L}_n$ of $(n-1)$-polyhedra for every $n=0,1,...$ which satisfy the following compatibility conditions:
\begin{itemize}
\item[1.] each member of $\mathcal{L}_n$ is a closed $\mathcal{L}_{n-1}$-manifold.
\item[2.] $S \mathcal{L}_{n-1} \subset \mathcal{L}_n$ (i.e. the suspension of an $(n-1)$-link is always an $n$-link).
\item[3.] $c \mathcal{L}_{n-1} \cap \mathcal{L}_n = \emptyset$ (i.e. the cone of an $(n-1)$-link is never an $n$-link).
\end{itemize}
Then an \textbf{$\mathcal{L}_n$-manifold with boundary} consists of a polyhedron whose links of vertices lie either in $\mathcal{L}_n$ or in $c \mathcal{L}_{n-1}$. The \textbf{boundary} consists of the subpolyhedron spanned by vertices whose links lie in the latter class.
\end{definition}

\begin{prop}
Let $W$ be an $\mathcal{L}_n$-manifold with boundary in a fixed theory with singularities $\mathcal{L}$. Then the boundary of $W$, denoted by $\partial W$, is well-defined and is itself a closed $\mathcal{L}_{n-1}$-manifold.
\end{prop}

\begin{proof}
By the third requirement of Definition \ref{theory}, the link of a vertex in $W$ is contained in $c \mathcal{L}_{n-1}$ if and only if it is not contained in $\mathcal{L}_n$. This shows that $\partial W$ is well-defined. For the second statement, note that $\partial W$ is a subpolyhedron by definition and that if $v$ is a vertex in $\partial W$ with $Lk(v,W)=cL$ for some $L \in \mathcal{L}_{n-1}$, then $Lk(v,\partial W) = L$.
\end{proof}

DEFINITION ÜBERARBEITEN: \\
LINKS STABIL BZGL SUBDIVISION? \\
RAND \\
PL POLYEDER STATT POLYEDER


\chapter{The basic sets $Q_i^{\overline{p}}$}
In this chapter we will define and study the so called "basic sets", originally introduced by M. Goresky and R. McPherson in \cite{GM}. Given a stratified PL pseudomanifold $X$ and a perversity $\overline{p}$, we construct subpolyhedra $Q_i^{\overline{p}}$ of $X$ for every $i \geq 0$. They are designed to give a connection between ordinary homology groups of these basic sets and the intersection homology groups of the whole space $X$.



\begin{thebibliography}[
\bibitem{pltopo}Colin Patrick Rourke, Brian Joseph Sanderson, \textit{Introduction to piecewise-linear topology.}
\bibitem{hatcher}Allen Hatcher, \textit{Algebraic Topology.} p.120.
\bibitem{lecturenotes}I. M. Singer and J. A. Thorpe, \textit{Lecture Notes on Elementary Topology and Geometry}, chapter 4.
\bibitem{GM}M. Goresky, R. McPherson, \textit{Intersection Homology Theory}.
\bibitem{BRS}S. Buoncristiano, C. Rourke and B. Sanderson, \textit{A Geometric Approach to Homology Theory}.
\end{thebibliography}

\end{document}
