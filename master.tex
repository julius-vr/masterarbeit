\documentclass{scrreprt}
\usepackage[english]{babel}
\usepackage[all]{xy}
\usepackage{hyperref}
\usepackage{amsthm}
\usepackage{amsmath}
\usepackage{amssymb}
\usepackage{tikz-cd}
\usepackage{mathtools}

\newtheorem{prop}{Proposition}[chapter]
\newtheorem{lemma}[prop]{Lemma}
\newtheorem{theorem}[prop]{Theorem}
\newtheorem{definition}[prop]{Definition}
\newtheorem{remark}[prop]{Remark}
\newtheorem{example}[prop]{Example}
\newtheorem{corollar}[prop]{Corollary}


\begin{document}

\tableofcontents

\chapter{Introduction}

Over the last century, ordinary homology has been shown to be an effective tool to study manifolds. One import reason for this is Poincar\'{e} duality, a remarkable symmetry which identifies homology groups of a closed, oriented manifold with its cohomology groups in complementary degrees. For example, it is this symmetry that enables one to define the signature, a bordism invariant which plays a key role in classification theory of manifolds. Beside manifold theory, a lot of mathematical research focuses on the study of singular spaces. These are topological spaces which are not necessarily manifolds, but that are not too far away from being manifolds. More precisely, 

\chapter{Some PL topology}
\label{sec:Einleitung}

In this chapter we collect some basic facts from the piecewise-linear category. The proofs are omitted and can be found in standard references like \cite{pltopo} and \cite{hatcher}. If the reader is already familiar with basic PL topology, this chapter may be skipped without loss of continuity.

\begin{definition}
Let $v_0,...,v_k \in \mathbb{R}^n$ be points in some affine space such that $\{ v_1-v_0,...,v_k-v_0 \}$ is a set of linearly independent vectors. We call

\begin{equation*}
[v_0,...,v_k] := \Biggl \{  \lambda_0v_0+...+ \lambda_kv_k \Bigg |  \sum_{i=0}^k \lambda_i =1 \text{ and } \lambda_i \geq 0 \text{ for all i} \Biggr \}
\end{equation*}
the simplex spanned by $\{v_0,...,v_k \}$. Its dimension is $k$ and we call it a \textbf{$k$-simplex} for short. The points that span a simplex are called vertices. For a simplex $\sigma$ we say that $\tau$ is a \textbf{face} of $\sigma$ if $\tau$ is a simplex spanned by a nonempty subset of the vertices of $\sigma$ and we abbreviate this by writing $\tau < \sigma$.
\end{definition}

\begin{definition}
A \textbf{simplicial complex} $K$ is a set of simplices that satisfies the following conditions:

\begin{itemize}
\item Every face of a simplex in $K$ is also contained in $K$.
\item The intersection of any two simplices $\sigma, \tau \in K$ is either empty or a face of both $\sigma$ and $\tau$.
\end{itemize}

We define the \textbf{polyhedron} of $K$ by $|K|:= \bigcup_{\sigma \in K} \sigma  $. A \textbf{subcomplex} of $K$ is a subset $L \subset K$ such that $L$ itself is a simplicial complex. The \textbf{p-skeleton} of $K$ is the subcomplex of $K$, given by $K_{(p)}= \{ \sigma \in  K | \text{dim}(\sigma) \leq p \}$. We denote the set of vertices of $K$ by $V(K)$. The \textbf{dimension} of $K$ is the largest dimension of any simplex contained in $K$. If no such maximum exists, we say that $K$ is of dimension $\infty$. $K$ is said to be of \textbf{pure dimension} $n$ if every simplex of $K$ is a face of some $n$-simplex in $K$.
\end{definition}


\begin{definition}
Let $K$ be a simplicial complex of pure dimension $n$. The \textbf{boundary} $Bd(K)$ of $K$ is defined to be the (possibly empty) subcomplex of $K$ of pure dimension $n-1$, whose $(n-1)$-simplices are those $(n-1)$-simplices of $K$ which are incident to precisely one $n$-simplex in $K$.
\end{definition}

\begin{definition}
For a simplicial complex $K$ and a simplex $\sigma \in K$, we call
\begin{equation*}
St(\sigma,K)= \{ \rho \in K | \rho < \tau, \sigma < \tau \text{ for some } \tau \in K \}  
\end{equation*}
the \textbf{star} of $\sigma$ in $K$. The \textbf{link} of $\sigma$ in $K$ is given by
\begin{equation*}
Lk(\sigma,K) = \{ \rho \in St(\sigma,K) | \sigma \cap \rho = \emptyset \}.
\end{equation*}
\end{definition}

\begin{definition}
Suppose that there are two simplicial complexes $K$ and $K'$ such that $|K| = |K'|$. If every simplex of $K'$ is contained in some simplex of $K$, we say that $K'$ is a \textbf{subdivision} of $K$ and write $K' \lhd K$.
\end{definition}


\begin{example}
Given a simplicial complex $K$ there is always an inductive process that yields a subdivision of $K$. Assume that $K_{(p-1)}$ has already been subdivided and let $\sigma=[v_0,...,v_p]$ be a $p$-simplex in $K$. The point $\hat{\sigma}=\frac{1}{p+1} \sum_{i=0}^p v_i$ lies in the interior of $\sigma$ and is called its \textbf{barycenter}. The \textbf{barycentric subdivision} of $\sigma$ is the decomposition of $\sigma$ into the $p$-simplices $[\hat{\sigma},w_0,...,w_{p-1}]$ where, inductively, $[w_o,...,w_{p-1}]$ is a $(p-1)$-simplex in the barycentric subdivision of a face $[v_0,...,\overline{v_i},...,v_p]$. (In this notation, the vertex $v_i$ is omitted.) Continuing this procedure for every $p$-simplex $\sigma$ leads to a decomposition of all simplices in $K_{(p)}$. The induction starts at $p=0$ where the barycentric subdivision of a $0$-simplex $[v_0]$ is just $[v_0]$ itself. It is guaranteed that this process leads to a subdivision $K^1 \lhd K$, called the \textbf{first barycentric subdivision} of $K$. For details, see \cite{hatcher}. More generally, the $r$-th barycentric subdivision is inductively given by $K^r:=(K^{r-1})^1$.
\end{example}


\begin{definition}
A topological space $X$ is said to be $\mathbf{triangulable}$ if there exists a simplicial complex $T$, together with a homeomorphism $\phi : |T| \to X$. The triple $(T,X, \phi)$ is called a \textbf{triangulation} of $X$. In this situation we will simply say that $T$ is a triangulation of $X$, by abuse of notation.
\end{definition}


\begin{definition}
A \textbf{PL space} is a pair $(X,\mathcal{T})$ consisting of a topological space $X$ and a class $\mathcal{T}$ of locally finite triangulations of $X$ which satisfies the following conditions:
\begin{itemize}
\item If $T \in \mathcal{T}$ then $T' \in \mathcal{T}$ for any subdivision $T' \lhd T$.
\item If $T,T' \in \mathcal{T}$ then there exists $T'' \in \mathcal{T}$ such that both $T'' \lhd T$ and $T'' \lhd T'$.
\end{itemize}
We will simply write $X$ for a PL space $(X,\mathcal{T})$ if there is no danger of confusion. A \textbf{closed PL subspace} of $X$ is a subcomplex of a suitable triangulation of $X$.
\end{definition}

\begin{definition}
Given simplicial complexes $K$ and $L$ we call a map $f: |K| \to |L|$ \textbf{simplicial} if $f$ maps each simplex of $K$ linearly onto some simplex of $L$. A map \\ $g: |K| \to |L|$ between PL spaces  is said to be a \textbf{PL map} if there exist subdivisions $K' \lhd K$ and $L' \lhd L$ such that $g: K' \to L' $ is simplicial.
\end{definition}
Note that a simplicial map $f: K \to L$ is given by linear extension of a (set-theoretic) function $V(K) \to V(L)$.

\begin{definition}
A \textbf{simplicial isomorphism} between two simplicial complexes $K$ and $L$ is given by a bijection $f : V(K) \to V(L)$ such that $[v_0,...,v_k]$ is a $k$-simplex in $K$ if and only if $[f(v_0),...,f(v_k)]$ is a $k$-simplex in $L$. In particular, extending $f$ linearly yields a homeomorphism between the underlying polyhedra $|K|$ and $|L|$. A map $g: |K| \to |L|$ between PL spaces is a \textbf{PL isomorphism} if there exist subdivisions $K' \lhd K$ and $L' \lhd L$ such that $g: |K'| \to |L'|$ is a simplicial isomorphism.
\end{definition}

\begin{prop}\label{imageplmappolyhedron}
Let $f: |K| \to |L|$ be a PL map, where $|K|$ is compact. Then $f(|K|)$ is a compact polyhedron.
\end{prop}

\begin{proof}
See \cite{pltopo}, chapter 2.
\end{proof}

\begin{theorem}(Simplicial Approximation Theorem).
Let $f: |K| \to |L|$ be a map between polyhedra. Then there exist subdivisions $K' \lhd K$ and $L' \lhd L$ and a simplicial map $g: |K'| \to |L'|$ which is $\epsilon$-homotopic to $f$, i.e. if $\epsilon: |L| \to (0,\infty)$ is a map, then there is a map $H: |K| \times [0,1] \to |L|$ with $H(|K| \times \{ 0\})=f$, $H(|K| \times \{ 1\})=g$ and diam$(H(\{ x \} \times [0,1])) < \epsilon(f(x))$ for all $x \in |K|$.
\end{theorem}

\begin{proof}
See \cite{lecturenotes}, chapter 4.
\end{proof}

\begin{definition}
Let $K,L$ be simplicial complexes. Their \textbf{join} $K*L$ is defined to be the simplicial complex with underlying vertex set $V(K*L):=V(K) \cup V(L)$ and whose simplices are of the form $[v_0,...,v_k,w_0,...,w_l]$ , where $[v_0,...,v_k] \in K, [w_0,...w_l] \in L$. In particular, we define $cK := pt. * K$ to be the \textbf{cone} on $K$ and we call $SK:=S^0 * K$ the \textbf{suspension} of $K$.
\end{definition}

\begin{prop}\label{subdivisionlink}
For a simplicial complex $K$, let $v \in K$ be a vertex and let $K' \lhd K$ be a subdivision. Then, $Lk(v,K)$ and $Lk(v,K')$ are PL isomorphic.
\end{prop}

\begin{proof}
See $\cite{hudson}$, p.23.
\end{proof}

\begin{definition}
Let $X$ be a PL space with triangulation $T$ and let $v \in X$ be a point. By subdividing $T$ if necessary, we may assume that $v$ is a vertex of $T$. We define $Lk(v,X)$ to be the PL isomorphism class of $Lk(v,T)$. If $T'$ is another triangulation of $X$ which contains $v$ as a vertex, then there is a triangulation $T''$ with $T'' \lhd T$ and $T'' \lhd T'$, and by Prop.\ref{subdivisionlink} we have
\begin{align*}
Lk(v,T) \cong Lk(v,T'') \cong Lk(v,T').
\end{align*}
Thus, the \textbf{link} $Lk(v,X)$ is well-defined.
\end{definition}

\begin{definition}
An n-dimensional \textbf{PL manifold} is a PL space $M$ such that the link of each point $x \in M$ is PL isomorphic to an $(n-1)$-dimensional sphere. An n-dimensional \textbf{PL manifold with boundary} is a PL space $W$ such that the link of each point $x \in W$ is PL isomorphic to either an $(n-1)$-dimensional sphere or to an $(n-1)$-dimensional ball.
\end{definition}



\chapter{A bordism approach to homology theories}\label{bordismhomology}
The purpose of this chapter is to give a geometric treatment of homology theories, as described by S. Buoncristiano, C. Rourke and B. Sanderson in \cite{BRS}. First we use local link properties to determine a certain class of polyhedra. Then, in an analogous manner to ordinary bordism theory, we define groups of bordism classes of maps, whose domains lie in the polyhedral class defined before. We observe that interesting examples of generalized homology theories can be interpreted in this way, including ordinary homology, ($\mathbb{Z}/2$)-homology and PL bordism theory. \newline
For the rest of this chapter we make the convention that every polyhedron is PL and of pure dimension.

\section{Orientations}

\begin{definition}Let $K$ be a simplicial complex and suppose that we have chosen an ordering of the underlying set of vertices.
\begin{itemize}
\item An \textbf{orientation} of a $0$-dimensional simplex $\sigma \in K$ is the choice of a function $\sigma \to \{ -1,1 \}$
\item For $k \geq 1$, let $\sigma = (v_0,...,v_k) \in K$ be a $k$-simplex. Every permutation $\pi$ on $\{ 0,...,k \}$ is given by a product of transpositions and the number of transpositions needed to obtain $\pi$ is uniquely determined by $\pi$. The parity of this number is called the \textbf{signature} of $\pi$. We call two permutations equivalent if and only if their signatures agree. Consequently, there are two equivalence classes of permutations: Those of even signature and those of odd signature. An \textbf{orientation} for $\sigma$ is the choice of such an equivalence class.
\end{itemize}
\end{definition}

\begin{remark}
Note that an orientation on a simplex $\sigma$ induces a canonical orientation on any face $\tau < \sigma$ by deletion of all the vertices of $\sigma$ that do not occur in $\tau$.
\end{remark}

\begin{definition}
Let $K$ be a simplicial complex of pure dimension $n$. We call $|K|$ \textbf{orientable} if it is possible assign an orientation to all the $n$-simplices in $K$ so that whenever two $n$-simplices $\sigma_1, \sigma_2$ have a common face $\tau$, the two orientations on $\tau$ which are induced by $\sigma_1$ and $\sigma_2$, are opposite. An \textbf{orientation} for $|K|$ is such a choice of orientations of all the $n$-simplices. It follows that if $|K|$ is orientable, then on each connected component there are precisely two possible orientations. If $|K|$ is oriented, we write $-|K|$ for the geometric realization of $K$ with reversed orientation.
\end{definition}

\section{Bordism theories}

\begin{definition}
Suppose we are given a class $\mathcal{L}_n$ of $(n-1)$-polyhedra which is closed under PL isomorphism. A \textbf{closed $\mathcal{L}_n$-manifold} is a polyhedron $M$ such that the link of each vertex of $M$ lies in $\mathcal{L}_n$. 
\end{definition}

\begin{definition}\label{theory}
A \textbf{theory with singularities} $\mathcal{L}$ consists of a class $\mathcal{L}_n$ of $(n-1)$-polyhedra for every $n=0,1,...$ which satisfy the following compatibility conditions:
\begin{itemize}
\item[1.] each member of $\mathcal{L}_n$ is a closed $\mathcal{L}_{n-1}$-manifold.
\item[2.] $S \mathcal{L}_{n-1} \subset \mathcal{L}_n$ (i.e. the suspension of an $(n-1)$-link is always an $n$-link).
\item[3.] $c \mathcal{L}_{n-1} \cap \mathcal{L}_n = \emptyset$ (i.e. the cone of an $(n-1)$-link is never an $n$-link).
\item[4.] $S(c\mathcal{L}_{n-2}), c(c \mathcal{L}_{n-2}) \subset c \mathcal{L}_{n-1}$.
\end{itemize}
Then an \textbf{$\mathcal{L}_n$-manifold with boundary} consists of a polyhedron whose links of vertices lie either in $\mathcal{L}_n$ or in $c \mathcal{L}_{n-1}$. The \textbf{boundary} consists of the subpolyhedron spanned by vertices whose links lie in the latter class.
\end{definition}

\begin{prop}
Let $W$ be an $\mathcal{L}_n$-manifold with boundary in a fixed theory with singularities $\mathcal{L}$. Then the boundary of $W$, denoted by $\partial W$, is well-defined and is itself a closed $\mathcal{L}_{n-1}$-manifold.
\end{prop}

\begin{proof}
By the third requirement of Def.\ref{theory}, the link of a vertex in $W$ is contained in $c \mathcal{L}_{n-1}$ if and only if it is not contained in $\mathcal{L}_n$. This shows that $\partial W$ is well-defined. For the second statement, note that $\partial W$ is a subpolyhedron by definition and that if $v$ is a vertex in $\partial W$ with $Lk(v,W)=cL$ for some $L \in \mathcal{L}_{n-1}$, then $Lk(v,\partial W) = L$.
\end{proof}

\begin{definition}\label{bordismusdef}
Let $\mathcal{L}$ be a theory with singularities and let $(X,A)$ be a pair of topological spaces (i.e. $A \subset X$ is a subspace). For two compact, oriented $\mathcal{L}_i$-manifolds (possibly with boundary) $P,Q$  assume that we are given continuous maps $f:(P, \partial P) \to (X,A)$ and $g:(Q , \partial Q) \to (X,A)$. An \textbf{oriented bordism} between $f$ and $g$ is a triple $(F,W,Z)$, where $W$ is a compact, oriented $\mathcal{L}_{i+1}$-manifold with boundary, s.t. $\partial W \cong P \sqcup -Q \cup Z$, $Z$ is a compact, oriented $\mathcal{L}_i$-manifold (possibly with boundary), s.t. $\partial Z\cong \partial P \sqcup - \partial Q$ and $F: (W,Z) \to (X,A)$ is a continuous map with $F _{|P} = f$ and $F_{|Q}=g$. If there exists a bordism between them, $f$ and $g$ are said to be \textbf{bordant} and we abbreviate this by writing $f \sim_{bord} g$. We call $F$ a \textbf{null-bordism} for $f$ if $Q= \emptyset$ and we say that $f$ is \textbf{null-bordant} if there exists a null-bordism for $f$.
\end{definition}

\begin{definition}\label{isomorphiedef}
With the notation as in the previous definition, $f: (P, \partial P) \to (X,A)$ and $g: (Q, \partial Q) \to (X,A)$ are called \textbf{isomorphic} if there exists a PL isomorphism $h : P \to Q$ s.t. the diagram \newline
\begin{xy}
  \xymatrix{
      (P, \partial P) \ar[rr]^h \ar[rd]_f  &     &  (Q, \partial Q) \ar[dl]^g  \\
                             &  (X,A)  &
  }
\end{xy}
\newline
commutes. Let $Isom_{i}^{\mathcal{L}}(X,A)$ be the set of isomorphism classes of maps $f: (P, \partial P) \to (X,A)$, where $P$ varies over all compact $\mathcal{L}_i$-manifolds of a fixed theory $\mathcal{L}$.
\end{definition}

\begin{prop}
The relation $\sim_{bord} $ is an equivalence relation on $Isom_{i}^{\mathcal{L}}(X,A)$.
\end{prop}

\begin{proof}
Let $[f: (P, \partial P) \to (X,A)] \in Isom_{i}^{\mathcal{L}}(X,A)$ and let $I=[0,1]$ denote the closed interval. The only links in $I$ are $S^0$ and $pt.$ and so there are four types of links in $P \times I$, namely $S^0 * L \cong SL,\  pt. * L \cong cL,\  S^0 * cL' \cong S(cL')$ and $pt. * cL' \cong c(cL')$, where $L \in \mathcal{L}_i$ and $L' \in \mathcal{L}_{i-1}$ are links of $P$. The second statement of Def. \ref{theory} ensures that the first type of links is contained in $\mathcal{L}_{i+1}$ and by the fourth statement all other types of links lie in $c \mathcal{L}_i$. There exists a canonical orientation of $P \times I$ s.t. $\partial (P \times I)= P \sqcup -P \cup \partial P \times I$. So, $P \times I$ is an oriented $\mathcal{L}_{i+1}$-manifold and we define $F: (P \times I, \partial(P \times I)) \to (X,A)$ by $F(x,s)=f(x)$ for every $s \in I$. Then $f \sim_{bord} f$ via $F$ and this bordism is well-defined on the isomorphism class of $f$, which shows reflexivity. Symmetry follows from the fact that disjoint union commutes, up to isomorphism. Now suppose $f \sim_{bord} g$ and $g \sim_{bord} h$ via bordisms $F$ and $G$, respectively, where $g: (Q, \partial Q) \to (X,A)$, $F: (W,Z) \to (X,A)$ and $G: (W',Z') \to (X,A)$. If $W \cup_{Q} W'$ denotes the space obtained by glueing $W$ and $W'$ along $Q$, we let $H: W \cup_{Q} W' \to X$ be the unique map with $H_{|W} = F$ and $H_{|W'}=G$. If $v$ is any vertex in $Q$, we have $Lk(v,W)=cLk(v,Q)=Lk(v,W')$, and therefore $Lk(v,W \cup_{Q} W') = SLk(v,Q)$, which lies in $\mathcal{L}_{i+1}$, by definition. Since the links of all other vertices remain unaltered, we see that $W \cup_{Q} W'$ is in fact an $\mathcal{L}_{i+1}$-manifold, which clearly is compact, as $W$ and $W'$ are. Moreover, we have $\partial (W \cup_{Q} W')= dom(f) \sqcup -dom(h) \cup (Z \cup_{\partial Q} Z')$ with $\partial (Z \cup_{\partial Q} Z')= \partial (dom(f)) \sqcup - \partial (dom(h))$, and the same argument as before shows that $Z \cup_{\partial Q} Z'$ is an $\mathcal{L}_i$-manifold. We conclude that $f \sim_{bord} h$ via $H: (W \cup_{Q} W',Z \cup_{\partial Q} Z') \to (X,A)$, and again $H$ is a well-defined bordism on isomorphism classes of $f$ and $h$. This proves transitivity and completes the proof.
\end{proof}

\begin{definition}\label{bordismgroup}
For a theory with singularities $\mathcal{L}$, we define  
\begin{equation*}
\Omega_i^\mathcal{L}(X,A):= Isom_i^\mathcal{L}(X,A)/ \sim_{bord}
\end{equation*}
 and call it the \textbf{$i$-th (relative) $\mathcal{L}$-bordism set} with base $(X,A)$. Moreover, we define
\begin{equation*}
\Omega_i^\mathcal{L}(X):=\Omega_i^\mathcal{L}(X, \emptyset)
\end{equation*}
and call it the \textbf{$i$-th (absolute) $\mathcal{L}$-bordism set} with base $X$.
\end{definition}

\begin{remark}
\begin{itemize}
\item[1.] There is an obvious notion of unoriented $\mathcal{L}$-bordism by removing all references to orientations. We denote the associated bordism sets by $\Omega_i^{\underline{\mathcal{L}}}(X,A)$.
\item[2.] For the absolute case, note that $\Omega_i^{\mathcal{L}}(X)$ consists of bordism classes of maps \\ $f: P \to X$, where $P$ is a, necessarily closed $\mathcal{L}_i$-manifold. For some $g: Q \to X$, the bordism relation between $f$ and $g$ reduces to the existence of some compact, oriented $\mathcal{L}_{i+1}$-manifold $W$ with $\partial W \cong P \sqcup -Q$ and a map $F: W \to X$ s.t. $F_{|P}=f$ and $F_{|Q}=g$.
\end{itemize}
\end{remark}

So far, we have not discussed interactions of two $\mathcal{L}$-manifolds with each other. For example, the product of two $\mathcal{L}_i$-manifolds is not an $\mathcal{L}_i$-manifold, in general. Howewer, the following is true:

\begin{prop}
$\Omega_i^{\mathcal{L}}(X,A)$ is an abelian group with respect to disjoint union.
\end{prop}

\begin{proof}
Let $[f: (P, \partial P) \to (X,A)], [g: (Q,\ \partial Q) \to (X,A)] \in \Omega_i^{\mathcal{L}}(X,A)$. Then $P \sqcup Q$ is an $\mathcal{L}_i$-manifold, as the links are unaltered by disjoint union, and so $f \sqcup g: (P \sqcup Q, \partial P \sqcup \partial Q) \to (X,A)$ represents a class in $\Omega_i^{\mathcal{L}}(X,A)$. Consequently, we let $[f] \sqcup [g]:= [f \sqcup g]$. Suppose that $f'$ and $g'$ are representatives of $[f]$ and $[g]$, respectively, via bordisms $F$ and $G$. Then $F \sqcup G$ is a bordism between $f \sqcup g$ and $f' \sqcup g'$, which shows that the above definition is a well-defined operation. Associativity and commutativity clearly hold. The identity element is given by the class of the empty map (and so by the class of any null-bordant map). If $[-f]$ denotes the class of $f$ with reversed orientation of its domain, we observe once again that $F: (P \times I, \partial(P \times I)) \to (X,A)$ with $F(x,s)=f(x)$ is a bordism between $f$ and $-f$. We conclude
\begin{equation*}
[f] \sqcup [-f] = [f \sqcup -f] = 0 ,
\end{equation*}
therefore the inverse of $[f]$ is given by $[-f]$. In the unoriented setting, we find that every bordism class is self-inverse.
\end{proof}

The previous Proposition allows us to speak of bordism groups, rather than of bordism sets and we adapt this terminology for the rest of this discussion. \newline So far, we did not study connections of $\mathcal{L}$-bordism groups for varying base spaces. This will be done in the following.

\begin{definition}
Let $\boldsymbol{CW^2}$ denote the category of pairs of finite CW complexes with continuous maps between them and let $\boldsymbol{Ab}$ be the category of abelian groups with group homomorphisms between them. Moreover, let $T$ denote the covariant functor on $\boldsymbol{CW^2}$, given by
\begin{align*}
&T(X,A) = (A, \emptyset) \text{, for any } (X,A) \in \boldsymbol{CW^2},  \\
&T(f) = f_{|(A, \emptyset)} : (A, \emptyset) \to (B, \emptyset) \text{, for any } f: (X,A) \to (Y,B) \in \boldsymbol{CW^2}.
\end{align*}
A \textbf{generalized homology theory} is a pair $(H_*, \partial_*)$, consisting of a sequence of functors $H_i : \boldsymbol{CW^2} \to \boldsymbol{Ab}$, together with a sequence of natural transformations \\ $\partial_i : H_i \to H_{i-1} \circ T$, s.t. the following three conditions are satisfied: \begin{itemize}
\item[1.] \textbf{Homotopy-invariance:} If $f,g \in \boldsymbol{CW^2}$ are homotopic maps, then $H_i(f)=H_i(g)$ for all $i$.
\item[2.] \textbf{Excision:} Let $(X,A) \in \boldsymbol{CW^2}$ and suppose $U$ is a subspace of $X$ with $\overline{U} \subset int(A)$, then the inclusion $j: (X-U,A-U) \to (X,A)$ induces isomorphisms 
\begin{equation*}
j_* : H_i(X-U,A-U) \to H_i(X,A),
\end{equation*}
for all $i$.
\item[3.] \textbf{Long exact sequence:} If $(X,A) \in \boldsymbol{CW^2}$ and if $j: (A, \emptyset) \to (X, \emptyset)$ and $k: (X, \emptyset) \to (X,A)$ denote the inclusions, the sequence
\begin{equation*}
\begin{xy}
\xymatrix{ 
\cdots \ar[r]^{}     &   H_{i+1}(X,A) \ar[r]^{\ \ \partial_{i+1}}    &   H_i(A) \ar[r]^{H_i(j)}   &  H_i(X) \\
  \ar[r]^{H_i(k)}   &   H_i(X,A)  \ar[r]^{\partial_i}   &   H_{i-1}(A) \ar[r]^{}   &   \cdots
}
\end{xy}
\end{equation*}
is exact, for all $i$.
\end{itemize}
A generalized homology theory $(H_*, \partial_*)$ is called \textbf{ordinary} if the following additional requirement holds:
\begin{itemize}
\item[4.] \textbf{Dimension:} $H_i(pt.)=0$ for $i \neq 0$. Then $H_0(pt.)$ is called the \textbf{coefficient group} of $H_*$.
\end{itemize}
\end{definition}

\begin{theorem}\label{Omegageneralizedhomology}
For a theory with singularities $\mathcal{L}$, the pair $(\Omega_*^{\mathcal{L}}, \partial_*)$ is a generalized homology theory, where 
\begin{align*}
\partial_i :\  \Omega_i^{\mathcal{L}}(X,A) &\to \Omega_{i-1}^{\mathcal{L}}(A) \\
[f: (P, \partial P) \to (X,A)] &\mapsto [f_{| \partial P}: \partial P \to A]
\end{align*}
for any pair $(X,A) \in \boldsymbol{CW^2}$.
\end{theorem}

\begin{proof}
First, if $\phi : (X,A) \to (Y,B) \in \boldsymbol{CW^2}$, then the induced map of $\phi $ is given by
\begin{align*}
\phi_* : \Omega_i^{\mathcal{L}}(X,A) &\to \Omega_i^{\mathcal{L}}(Y,B) \\
[f] &\mapsto [\phi \circ f],
\end{align*}
which is well-defined since if $F$ is a bordism over $(X,A)$, then $\phi \circ F$ is a bordism over $(Y,B)$. Moreover, this construction respects the corresponding group structures, behaves functorial and we have $(id)_*=id$. Similarly, $\partial_i$ is well-defined, since if $F:(W,Z) \to (X,A)$ is a bordism between $f$ and $g$ over $(X,A)$, then $F_{|Z}$ is a bordism between $f_{| \partial dom(f)}$ and $g_{| \partial dom(g)}$ over $A$. The maps $\partial_i$ are natural, as composition of maps commutes with restriction to the boundary. 
\begin{itemize}
\item[1.] Homotopy-invariance: Let $\phi, \psi : (X,A) \to (Y,B)$ be two homotopic maps via a homotopy $H$. For $[f] \in \Omega_i^{\mathcal{L}}(X,A)$, the map
\begin{equation*}
F: (dom(f) \times I) \to (X,A) 
\end{equation*}
with $F(x,s)=H(f(x),s)$ is a bordism between $\phi \circ f$ and $\psi \circ f$ (with reversed orientation of its domain). We therefore have
\begin{align*}
\phi_* ([f]) = [\phi \circ f] = [\psi \circ f] = \psi_* ([f])
\end{align*}
and since $[f]$ was chosen arbitrarily, we conclude $\phi_*= \psi_*$.
\item[2.] Excision: For $(X,A) \in \boldsymbol{CW^2}$ and $U \subset A$ with $\overline{U} \subset int(A)$, let 
\begin{equation*}
j_i: \Omega_i^{\mathcal{L}}(X-U, A-U) \to \Omega_i^{\mathcal{L}}(X,A)
\end{equation*}
denote the map induced by inclusion. We show that $j_i$ is an isomorphism. To see surjectivity, consider $[f: (P, \partial P) \to (X,A)] \in \Omega_i^{\mathcal{L}}(X,A)$ and let $U_1 = f^{-1}(U)$ and $A_1= f^{-1}(A)$. We choose a triangulation $T$ of $P$, fine enough such that the smallest subcomplex of $T$ which contains every simplex that meets $P-A_1$, is contained in $P-U_1$. This is possible, since $d(P-int(A_1), \overline{U_1})>0$ for any metric $d$ on $P$. If we denote this subcomplex by $K$, note that for any vertex $v \in K$ either $Lk(v,K)=Lk(v,T)$ or, by the first condition of Def. \ref{theory}, there exists a vertex $w \in L:= Lk(v,T)$ such that $Lk(w,L)=Lk(v,Bd(K))$. It then follows that
\begin{align*}
Lk(v,K) = w * Lk(v,Bd(K)) = w * Lk(w,L) \cong c(Lk(w,L)),
\end{align*}
and since $Lk(w,L) \in \mathcal{L}_{i-1}$, $|K|$ is in fact an $\mathcal{L}_i$-manifold. By construction, \\ $f_1:=f|_{|K|}$ defines a class in $\Omega_i^{\mathcal{L}}(X-U,A-U)$ and 
\begin{align*}
F: \frac{P \times I}{ (P-|K|) \times \{ 1 \} } &\to (X,A) \\
(x,s) &\mapsto f(x)
\end{align*}
defines a bordism between $f$ and $f_1$ over $(X,A)$. Consequently, we have \\ $j_i ([f_1]) = [f]$. For injectivity, suppose $j_i([f])=0$ and let $F$ be the corresponding null-bordism for $f$ over $(X,A)$. Then, the same construction as before applied to $F$ provides a null-bordism $F_1$ for $f$ over $(X-U,A-U)$, which shows $[f]=0$.
\item[3.] Long exact sequence: For $i \geq 0$, let \\
\begin{tikzcd}
\cdots \arrow[r] & \Omega_i^{\mathcal{L}}(A) \arrow[r,"j_i"] & \Omega_i^{\mathcal{L}}(X) \arrow[r,"k_i"] & \Omega_i^{\mathcal{L}}(X,A) \\ 
& \arrow[r,"\partial_i"] & \Omega_{i-1}^{\mathcal{L}}(A) \arrow[r,"j_{i-1}"] & \Omega_{i-1}^{\mathcal{L}}(X) \arrow[r]& \cdots
\end{tikzcd}
\\
be the sequence of the induced maps of inclusions $j,k$ and restriction map $\partial_i$. \\
{$im(j_i) = ker(k_i):$} If $[f:P \to A] \in \Omega_i^{\mathcal{L}}(A)$, then $F: (P \times I, -P) \to (X,A)$ with $F(x,s)=f(x)$ is a null-bordism for $f$ over $(X,A)$, which shows $k_i \circ j_i = 0$. Now, suppose that $[g: Q \to X] \in \Omega_i^{\mathcal{L}}(X)$ with $k_i([g])=0$. This means that there exists a null-bordism $G: (W,Z) \to (X,A)$ for $g$ over $(X,A)$. We then have 
\begin{align*}
j_i([G_{|Z}: -Z \to A]) = [g],
\end{align*}
as $G: W \to X$ is a bordism between $G_{|Z}$ and $g$ over $X$. \\
$im(k_i)=ker(\partial_i):$ Since $\partial_i \circ k_i$ is given by restriction to an empty boundary, $\partial_i \circ k_i=0$ is obvious. On the other hand, if 
\begin{align*}
\partial_i([f:(P, \partial P) \to (X,A)])= [f_{| \partial P} : \partial P \to A] = 0,
\end{align*}
there exists a null-bordism $F: W \to A$ for $f_{| \partial P}$ over $A$. In particular, we have $\partial W= \partial P$. Let $Q$ denote the cylinder $P \times I$, in which we add $W$ by glueing it along the boundary of $P \times \{ 0 \}$. Then $Q$ is an $\mathcal{L}_{i+1}$-manifold with $P$ and $P \cup_{\partial P} W$ sitting inside its boundary. We define the map $G: Q \to X$ to be $f$ on every level of the cylinder and to be $F$ on $W$. The condition $F_{| \partial W}=f_{| \partial P}$ ensures that $G$ is well-defined and continuous. Moreover, $G$ is a bordism between $G_{|P \cup_{\partial P} W}$ and $f$ over $(X,A)$ and since $P \cup_{\partial P} W$ has no boundary, we conclude 
\begin{align*}
k_i([G_{|P \cup_{\partial P} W}])=[f].
\end{align*}
$im(\partial_i)= ker(j_{i-1}):$ If $[f:(P, \partial P) \to (X,A)] \in \Omega_{i}^{\mathcal{L}}(X,A)$, we have 
\begin{align*}
j_{i-1} \circ \partial_i ([f])= [f_{| \partial P}: \partial P \to X] = 0,
\end{align*}
as a null-bordism over $X$ is given by $f$. For the other implication, consider $[g: Q \to A] \in \Omega_{i-1}^{\mathcal{L}}(A)$ with $j_{i-1}([g])=0$. If $G: W \to X$ is a corresponding null-bordism over $X$, it follows that
\begin{align*}
\partial_i([G: (W,Q) \to (X,A)]) = [g],
\end{align*}
\end{itemize}
which shows exactness of the sequence and completes the proof.
\end{proof}

Next, we give some examples of classes $\mathcal{L}$ that determine the corresponding bordism theory $\Omega_*^{\mathcal{L}}$.

\begin{example}
Consider the class of links $\mathcal{L}$, given by $\mathcal{L}_0 := \{ \emptyset \}$, $\mathcal{L}_n:= \{ X | X \cong S^{n-1} \}$ for $n \geq 1$. Then a closed $\mathcal{L}_n$-manifold is a polyhedron $M$ of dimension $n$, in which every vertex $v$ has a PL sphere as its link. Therefore $M$ is a closed PL manifold, since a small open neighborhood of $v$ is PL isomorphic to an open PL ball. Similarly, an $\mathcal{L}_n$-manifold with boundary is simply an $n$-dimensional PL manifold with boundary, and so the associated theory $\Omega_*^{\mathcal{L}}$ is "ordinary" (oriented) PL bordism theory, denoted by $\Omega_*^{PL}$.
\end{example}

\begin{example}\label{ordinaryhomologyexample}
We define a class of links $\mathcal{L}$ as follows: $\mathcal{L}_0 := \{ \emptyset \}$, $\mathcal{L}_1 := \{ X|X \cong S^0 \}$, and for $n \geq 2$ we let $\mathcal{L}_n$ be the class of \textbf{all} closed $\mathcal{L}_{n-1}$-manifolds. We would like to compute the coefficient group $\Omega_*^{\mathcal{L}}(pt.)$. For this, note that an $\mathcal{L}_0$-manifold is a disjoint union of points and an $\mathcal{L}_1$-manifold with boundary is a disjoint union of (closed) intervals and circles. This means that a point does not bound in $\mathcal{L}$, and so generates 
$\Omega_0^{\mathcal{L}}(pt.) \cong \mathbb{Z}.$ Moreover, if $P$ is a closed $\mathcal{L}_n$-manifold for $n \geq 1$, then $cP$ is an $\mathcal{L}_{n+1}$-manifold with boundary. Indeed, let $c$ denote the cone point. Then $Lk(c,cP)=P \in \mathcal{L}_{n+1}$ and if $v$ denotes some other vertex of $cP$, then $v$ lies in $P$ and if $L:=Lk(v,P)$, we have $Lk(v,cP)=cL$, which is the cone on a link in $\mathcal{L}_n$. This also shows, that the boundary of $cP$ consists of $P$, and so $P$ bounds in $\mathcal{L}$. We conclude that $\Omega_n^{\mathcal{L}}(pt.)=0$ and denote the corresponding theory by $\Omega_*^{ord}$. Note that in the unoriented setting of the same class, the higher coefficient groups still vanish. But in difference to the oriented case, we have $\Omega_0^{\underline{\mathcal{L}}}(pt.)= \mathbb{Z}/2$, since the disjoint union of two points is then the boundary of an interval. We denote the unoriented theory associated to $\mathcal{L}$ by $\Omega_*^{ord}(-; \mathbb{Z}/2)$. As the previous two generalized homology theories additionally satisfy the dimension axiom, they represent ordinary homology theory with $\mathbb{Z}$- and $\mathbb{Z}/2$-coefficients, respectively.
\end{example}

\chapter{The basic sets $Q_i^{\overline{p}}$}\label{basicsetschapter}
In this chapter we define and study the so-called "basic sets", originally introduced by M. Goresky and R. McPherson in \cite{GM}. Given a PL stratified pseudomanifold $X$ and a perversity $\overline{p}$, we construct subpolyhedra $Q_i^{\overline{p}}$ of $X$ for every $i \geq 0$. They are designed to give a connection between ordinary homology groups of these basic sets and the intersection homology groups of the whole space $X$. \\
We start with a brief introduction to PL intersection homology theory.

\section{Intersection homology}

\begin{definition}
A \textbf{0-dimensional PL stratified pseudomanifold} is a countable set of points with the discrete topology. An \textbf{n-dimensional PL stratified pseudomanifold} $X$ is a PL space together with a filtration of closed PL subspaces
\begin{equation*}
X=X_n \supset X_{n-1} = X_{n-2} \supset ... \supset X_0 \supset X_{-1} = \emptyset
\end{equation*}
such that the following conditions are satisfied:
\begin{itemize}
\item $X_{n-k} - X_{n-k-1}$ is a (possibly empty) PL manifold of dimension $n-k$, for each $k$.
\item $X-X_{n-2}$ is dense in $X$.
\item \textbf{Local normal triviality:} For every point $x \in X_{n-k} - X_{n-k-1}$ there exists an open neighborhood $U$ of $x$  in $X$ and a compact PL stratified pseudomanifold $L$ of dimension $k-1$ with filtration
\begin{equation*}
L = L_{k-1} \supset L_{k-3} \supset ... \supset L_0 \supset L_{-1}= \emptyset
\end{equation*}
and a PL isomorphism
\begin{equation*}
\phi : U \to \mathbb{R}^{n-k} \times c^{\circ} L
\end{equation*}
(where $c^{\circ}$ denotes the open cone), which restricts to PL isomorphisms $\phi_| : U \cap X_{n-l} \to \mathbb{R}^{n-k} \times c^{\circ} L_{k-l-1}$. We say that $\phi$ is \textbf{stratum-preserving}.
\end{itemize}
A closed subset $X_{n-k}$ occuring in the filtration of $X$ is called \textbf{stratum} of codimension $k$. We call $X_{n-k}-X_{n-k-1}$ the \textbf{pure stratum} of codimension $k$.
\end{definition}

\begin{definition}
Let $X$ be a PL space and $T$ be an admissible triangulation for $X$. Let $C_i^{T}(X)$ denote the free abelian group, generated by all ordered $i$-simplices of $T$. Suppose $T' \lhd T$ is a subdivision. If $\xi \in C_i^T(X)$, we can assign a canonical element $\xi' \in C_i^{T'}(X)$ by mapping each generator $\sigma \in C_i^T(X)$ to $\sum_{\sigma' \in T' , \sigma' \subset \sigma} \sigma' $ and extending linearly.
This yields a map
\begin{align*}
C_i^T(X) \to C_i^{T'}(X),
\end{align*}
which we call the canonical map. Now, define
\begin{align*}
C_i(X) := colim \ C_i^T(X),
\end{align*}
where the colimit ranges over all triangulations of the PL structure of $X$, with respect to the canonical maps. In other words, $C_i(X)$ consists of equivalence classes, represented by elements $\xi \in C_i^T(X)$, where $\xi$ and $\xi' \in C_i^{T'}(X)$ are equivalent if there is a common admissible subdivision $T''$ of $T$ and $T'$, such that the images of $\xi$ and $\xi'$ under the canonical maps coincide in $C_i^{T''}(X)$. For any $i$, the simplicial boundary maps (see. \cite{hatcher}, ch.2)
\begin{align*}
\partial_i^T: C_i^T(X) \to C_{i-1}^T(X)
\end{align*}
give rise to boundary maps 
\begin{align*}
\partial_i : C_i(X) \to C_{i-1}(X),
\end{align*}
which satisfy $\partial_i \circ \partial_{i-1}=0$. The associated homology groups
\begin{align*}
H_i(X):= H_i(C_*(X))
\end{align*}
are called \textbf{PL homology} groups of $X$.
\end{definition}

\begin{definition}
A \textbf{perversity} is a function $\overline{p}: \mathbb{N}_{\geq 2} \to \mathbb{N}$ such that:
\begin{itemize}
\item $\overline{p}(2)=0$,
\item $\overline{p}(k) \leq \overline{p}(k+1) \leq \overline{p}(k)+1$.
\end{itemize}
Given two perversities $\overline{p}, \overline{q}$, we write $\overline{p} \leq \overline{q}$ if $\overline{p}(k) \leq \overline{q}(k)$ for every $k$.
\end{definition}

\begin{example}
There are at least four important perversities:
\begin{itemize}
\item The \textbf{zero perversity} $\overline{0}$, defined as $\overline{0}(k)=0$ for all $k$.
\item The \textbf{lower-middle perversity} $\overline{m}= \{ 0,0,1,1,2,2,3,3,... \}$
\item The \textbf{upper-middle perversity} $\overline{n}= \{ 0,1,1,2,2,3,3,... \}$
\item The \textbf{top perversity} $\overline{t}$, given by $\overline{t}(k)=k-2$ for all $k$.
\end{itemize}
Two perversities $\overline{p}, \overline{q}$ are said to be \textbf{complementary} if $\overline{p}+ \overline{q}= \overline{t}$.
\end{example}


For the rest of this chapter, let $X$ be a fixed PL stratified pseudomanifold of dimension $n$ with strata $X_{n-k}$  and let $\overline{p}$ be a perversity.

\begin{definition}
A subspace $Y \subset X$ is said to be \textbf{$(\overline{p},i)$-allowable} if $dim(Y) \leq i$ and $dim(Y \cap X_{n-k}) \leq i+(n-k)-n+ \overline{p}(k)=i-k+\overline{p}(k)$, for all $k \geq 2$.
\end{definition}

\begin{definition}
For a triangulation $T$ of $X$ and $\xi \in C_i^T(X)$ let $|\xi|$ denote the \textbf{support} of $\xi$, i.e. the union of those simplices in $T$, which have non-zero coefficients in $\xi$. Suppose $T' \lhd T$ is an admissible subdivision of $T$ and $\xi' \in C_i^{T'}(X)$ is the image of $\xi$ under the canonical map. Then $|\xi|=|\xi'|$, and therefore any element $\alpha \in C_i(X)$ has a well-defined support $|\alpha|$.
\end{definition}

\begin{definition}
Let $IC_i^{\overline{p}}(X)$ be the subgroup of $C_i(X)$ which is generated by those chains $\xi$ such that $|\xi|$ is $(\overline{p},i)$-allowable and $|\partial \xi|$ is $(\overline{p},i-1)$-allowable. Then the boundary maps of the complex $C_*(X)$ restrict to boundary maps
\begin{align*}
\partial_i : IC_i^{\overline{p}}(X) \to IC_{i-1}^{\overline{p}}(X)
\end{align*}
by the constraint on the boundaries of elements in $IC_i^{\overline{p}}(X)$. Therefore, we have a chain complex $(IC_*^{\overline{p}}(X),\partial_*)$, and we define
\begin{align*}
IH_i^{\overline{p}}(X):=H_i(IC_*^{\overline{p}}(X))
\end{align*}
to be the $i$-th \textbf{intersection homology} group of $X$ for perversity $\overline{p}$.
\end{definition}

\section{Basic sets}

If not otherwise stated, $T$ denotes a fixed admissible triangulation for the PL stratified pseudomanifold $X^n$. By $T^1$, we denote the first barycentric subdivision of $T$ and the $p$-skeleton of $T$ is denoted by $T_{(p)}$, as usual.

\begin{definition}
For each $i \geq 0$ and fixed perversity $\overline{p}$ we define a function $L_i^{\overline{p}}: \{ 0,...,n+1 \} \to \mathbb{N}$ as follows: 
\begin{align*}
L_i^{\overline{p}}(0)=i,\ \ L_i^{\overline{p}}(1)=i-1,\ \ L_i^{\overline{p}}(n+1)=-1,
\end{align*}
and for $2 \leq c \leq n$ we let
\begin{align*}
L_i^{\overline{p}}(c)=
\begin{cases}
-1 &\text{ if $i-c+p(c) \leq -1$}  \\
n-c &\text{ if $i-c+p(c) \geq n-c$ } \\
i-c+p(c) &\text{ otherwise.}
\end{cases}
\end{align*}
Furthermore, let $\Delta_i^{\overline{p}}(c)=L_i^{\overline{p}}(c) - L_i^{\overline{p}}(c+1)$. Then the \textbf{i-th basic set} $Q_i^{\overline{p}}$ of $X$ with respect to $T$ is the subcomplex of $T^1$, which is spanned by the following set of barycenters of simplices in $T$:
\begin{align*}
\{ \hat{\sigma} | \sigma \in T,\ \Delta_i^{\overline{p}}(n-dim(\sigma))=1\}
\end{align*}
\end{definition}

From now on we will consider basic sets $Q_i^{\overline{p}}$ with respect to a fixed triangulation $T$ of $X$ without further mention.

\begin{remark}\label{remarkbasic}
\begin{itemize}
\item[1.] Note that $L_i^{\overline{p}}(c)$ represents the largest possible dimension of intersection of any $(\overline{p},i)$-allowable set with $X_{n-c}$.
\item[2.] Using the perversity restriction $\overline{p}(c) \leq \overline{p}(c+1) \leq \overline{p}(c)+1$, a simple case distinction shows that $\Delta_i^{\overline{p}}(c)$ is either $0$ or $1$.
\item[3.] A similar argument shows that if $\Delta_i^{\overline{p}}(c)=1$, then $\Delta_{i+1}^{\overline{p}}(c)=1$. We conclude that $Q_i^{\overline{p}}$ is a subcomplex of $Q_{i+1}^{\overline{p}}$.
\end{itemize}
\end{remark}

\begin{prop}\label{dimensionbasicset}
For any $k$-simplex $\sigma \in T$,  we have 
\begin{align*}
dim(Q_i^{\overline{p}} \cap \sigma)= L_i^{\overline{p}}(n-k).
\end{align*}
\end{prop}

\begin{proof}
If $\sigma^1$ denotes the first barycentric subdivision of $\sigma$, then $Q_i^{\overline{p}} \cap \sigma$ is a subcomplex of $\sigma^1$, spanned by barycenters of faces $\tau < \sigma$ with $\Delta_i^{\overline{p}}(n-dim(\tau))=1$. If $\tau_0 < \tau_1 < ... < \tau_{k-1} < \tau_k:= \sigma$ is a sequence of faces, where each $\tau_j$ is a $j$-simplex, then under consideration of Rem.\ref{remarkbasic}.2. a top-dimensional simplex in $Q_i^{\overline{p}} \cap \sigma$ is spanned by 
\begin{align*}
\sum_{j=0}^k \Delta_i^{\overline{p}}(n-dim(\tau_j)) = \sum_{j=0}^k \Delta_i^{\overline{p}}(n-j &)= L_i^{\overline{p}}(n-k) - L_i^{\overline{p}}(n+1) \\ &=L_i^{\overline{p}}(n-k) +1
\end{align*}
vertices, and we conclude that $dim(Q_i^{\overline{p}} \cap \sigma)= L_i^{\overline{p}}(n-k).$
\end{proof}

\begin{corollar}
$Q_i^{\overline{p}}$ is of dimension $i$.
\end{corollar}

\begin{proof}
For any $n$-simplex $\sigma \in T$, we have $dim(Q_i^{\overline{p}} \cap \sigma)=L_i^{\overline{p}}(0)=i$.
\end{proof}

\begin{lemma}\label{lemmabasicsets}
For complementary perversities $\overline{p}$ and $\overline{q}$, the equation
\begin{align*}
L_i^{\overline{p}}(c) + L_{n-i+1}^{\overline{q}}(c)=n-c-1
\end{align*}
holds for $2 \leq c \leq n+1$.
\end{lemma}

\begin{proof}
We begin with the case $-1 \leq i-c + \overline{p}(c) \leq n-c$. Using $\overline{p}(c)+ \overline{q}(c)=c-2$, we observe that this holds if and only if $-1 \leq n-i+1-c + \overline{q}(c) \leq n-c$, and so in this case we obtain
\begin{align*}
L_i^{\overline{p}}(c) + L_{n-i+1}^{\overline{q}}(c)=i-c+ \overline{p}(c)+ n-i+1-c+ \overline{q}(c)   =n-c-1.
\end{align*}
Now assume $i-c + \overline{p}(c) \leq -1$. Once again, we use the perversity constraint and see that this holds if and only if $n-i+1-c+ \overline{q}(c) \geq n-c$. We then have
\begin{align*}
L_i^{\overline{p}}(c) + L_{n-i+1}^{\overline{q}}(c)=-1 + n-c =n-c-1.
\end{align*}
Now the last case follows by symmetry, as $n-(n-i+1)+1=i$.
\end{proof}

\begin{prop}\label{retractions}
For $i \geq 1$ and complementary perversities $\overline{p}$ and $\overline{q}$ there exist simplex-preserving strong deformation retractions
\begin{align*}
&X-(Q_{n-i+1}^{\overline{q}} \cap |T_{(n-2)}|) \to Q_i^{\overline{p}} , \\
&X- Q_{n-i+1}^{\overline{q}} \to Q_i^{\overline{p}} \cap |T_{(n-2)}|.
\end{align*}
\end{prop}

\begin{proof}
For any $i \geq 1$ we have $\Delta_i^{\overline{p}}(0)=\Delta_i^{\overline{p}}(1)=1$, since $\overline{p}(2)=0$. This means that the barycenter $\hat{\sigma}$ of $\sigma$ is a vertex of $Q_i^{\overline{p}}$, whenever $dim(\sigma) \geq n-1$. If, on the other hand, $2 \leq c \leq n+1$, then
\begin{align*}
\Delta_i^{\overline{p}}(c)+ \Delta_{n-i+1}^{\overline{q}}(c)=1,
\end{align*}
by Lemma \ref{lemmabasicsets}. Thus, if $dim(\sigma) \leq n-2$, then $\hat{\sigma}$ is a vertex in precisely one of $Q_i^{\overline{p}}$ and $Q_{n-i+1}^{\overline{q}}$. We conclude that the set of vertices in $T^1$ which span $Q_{n-i+1}^{\overline{q}} \cap |T_{(n-2)}|$ is exactly the complement of the set of vertices that span $Q_i^{\overline{p}}$. Therefore, every simplex of $T^1$ is the join of its intersection with $Q_i^{\overline{p}}$, and of its intersection with $Q_{n-i+1}^{\overline{q}} \cap |T_{(n-2)}|$. Then the first retraction is given in each simplex of $T^1$ by retracting along these join lines. For the second retraction, observe that the above argument also shows that the set of vertices which span $Q_{n-i+1}^{\overline{q}}$ is the complement of the set of vertices which span $Q_i^{\overline{p}} \cap |T_{(n-2)}|$, and then use the same construction as before.
\end{proof}

\begin{definition}
A triangulation $T$ of a PL stratified pseudomanifold $X$ is called \textbf{subordinate to the stratification} of $X$ if each stratum $X_k$ is a subcomplex of $T$.
\end{definition}

\begin{remark}
Note that triangulations subordinate to the stratification of $X$ necessarily exist in our setting, since the strata $X_k$ are closed PL subspaces and since two admissible triangulations always have a common admissible subdivision. Even more, for two stratifications of $X$ there exists a triangulation subordinate to both.
\end{remark}

One important reason for our interest in the basic sets is the following geometric result.

\begin{prop}\label{basicsetsallowable}
Assume that $T$ is a triangulation subordinate to the stratification of $X$. Let $Q_i^{\overline{p}}$
denote the basic sets with respect to this triangulation. Then $Q_i^{\overline{p}}$ is $(\overline{p},i)$-allowable.
\end{prop}

\begin{proof}
Let $\sigma \subset X_{n-k}$ be a simplex. We know that $\sigma \in T$, since $X_{n-k}$ is a subcomplex of $T$. Then either $Q_i^{\overline{p}} \cap \sigma= \emptyset$ or, in view of Prop.\ref{dimensionbasicset},
\begin{align*}
dim(Q_i^{\overline{p}} \cap X_{n-k} \cap \sigma) &=dim(Q_i^{\overline{p}} \cap \sigma)=L_i^{\overline{p}}(n-dim(\sigma)) \\ &\leq i - (n- dim(\sigma))+ \overline{p}(n-dim(\sigma)) \\ &\leq i-k+ \overline{p}(k),
\end{align*}
where the first inequality holds under the assumption that $Q_i^{\overline{p}} \cap \sigma \neq \emptyset$ and the second inequality is ensured by $dim(\sigma) \leq n-k$ and by the perversity constraint $\overline{p}(c+1) \leq \overline{p}(c)+1$. Maximizing over all simplices $\sigma \in X_{n-k}$ shows
$dim(Q_i^{\overline{p}} \cap X_{n-k} ) \leq i-k+ \overline{p}(k).$
\end{proof}

Now suppose that the basic sets $Q_i^{\overline{p}}$ are constructed with respect to a triangulation $T$ subordinate to the stratification of $X$. Let 
\begin{align*}
H_i(Q_{i}^{\overline{p}}) \to H_i(Q_{i+1}^{\overline{p}})
\end{align*}
be the map in PL homology which is induced by the inclusion $Q_i^{\overline{p}} \subset Q_{i+1}^{\overline{p}}$. Under consideration of naturality of the induced chain morphism $C_*(Q_i^{\overline{p}}) \to C_*(Q_{i+1}^{\overline{p}})$, we see that an element $\alpha \in Im(H_i(Q_{i}^{\overline{p}}) \to H_i(Q_{i+1}^{\overline{p}}))$ is represented by $\sigma+ \partial \eta$, where $\sigma \in C_i(Q_i^{\overline{p}})$ is a cycle and $\eta \in  C_{i+1}(Q_{i+1}^{\overline{p}})$
is some chain with $\partial \eta \in C_i(Q_i^{\overline{p}})$. By Prop.$\ref{basicsetsallowable}$, this means that $\sigma$ is a cycle in $IC_i^{\overline{p}}(X)$ and $\eta \in IC_{i+1}^{\overline{p}}(X)$, and therefore $\alpha$ defines an intersection homology class. Thus, we have a well-defined homomorphism
\begin{align*}
\Psi: Im(H_i(Q_{i}^{\overline{p}}) \to H_i(Q_{i+1}^{\overline{p}})) \to IH_i^{\overline{p}}(X).
\end{align*}

\begin{theorem}\label{theorembasicsets}
In the setting of the discussion above, the map
\begin{align*}
\Psi: Im(H_i(Q_{i}^{\overline{p}}) \to H_i(Q_{i+1}^{\overline{p}})) \to IH_i^{\overline{p}}(X)
\end{align*}
is an isomorphism for any $i \geq 0$.
\end{theorem}

\begin{proof}
See \cite{GM}, p.148. The proof makes use of the retractions from Prop.\ref{retractions}. We omit it as we use a similar technique later.
\end{proof}

\begin{corollar}\label{IHindependentstrat}
The groups $IH_i^{\overline{p}}(X)$ are independent of the stratification of $X$.
\end{corollar}

\begin{proof}
For two stratifications of $X$, there is a triangulation $T$ of $X$ subordinate to both. If we construct the basic sets with respect to $T$, the previous theorem applies and shows that for either stratification, $IH_i^{\overline{p}}(X)$ is isomorphic to $Im(H_i(Q_{i}^{\overline{p}}) \to H_i(Q_{i+1}^{\overline{p}}))$.
\end{proof}

\begin{corollar}\label{IHindependenttriang}
If $T$ is an arbitrary triangulation of $X$ and if the $Q_i^{\overline{p}}$ are defined with respect to $T$, then
\begin{align*}
IH_i^{\overline{p}}(X) \cong Im(H_i(Q_{i}^{\overline{p}}) \to H_i(Q_{i+1}^{\overline{p}})).
\end{align*}
\end{corollar}

\begin{proof}
The skeleta of $T$ provide a stratification
\begin{align*}
X= |T_{(n)}| \supset |T_{(n-2)}| \supset |T_{(n-3)}| \supset \cdots \supset |T_{(0)}|
\end{align*}
for $X$, and clearly $T$ is subordinate to this stratification. Now Cor.$\ref{IHindependentstrat}$ gives the desired result.
\end{proof}

Note that the connection of ordinary homology and intersection homology that appears in Thm.\ref{theorembasicsets} gives another interpretation of intersection homology without making use of the intersection chain complex. The downside is the complicated construction of the basic sets. Nevertheless, this seems to be a good starting point to implement generalized intersection homology theories.

\chapter{A bordism approach to intersection homology}\label{IOmega}
In this chapter we impose certain admissibility conditions on maps whose codomain is a fixed base pseudomanifold. By additionally imposing these admissibility conditions on bordisms, we observe that such admissible bordism classes form an abelian group, which we call $\overline{p}$-allowable bordism, and it turns out that we obtain a bordism-type description of intersection homology in this way. The core ideas of this chapter stem from \cite{comezana}. \\
Throughout this chapter, $X$ denotes a PL stratified pseudomanifold of dimension $n$ and $\overline{p}$ denotes a fixed perversity, unless stated otherwise. The $k$-th stratum of $X$ is denoted by $X_k$, as usual. Moreover, we denote the pure strata of $X$ by $\mathcal{X}_k:= X_k - X_{k-1}$.

\section{General position}

\begin{definition}
Assume that $A$ and $B$ are two PL subspaces of an $n$-dimensional PL manifold $M$. We say that $A$ and $B$ are in \textbf{general position} in $M$ if 
\begin{align*}
dim(A \cap B) \leq dim(A) + dim(B) - n.
\end{align*}
\end{definition}

\begin{definition}
Let $A,B \subset X$ be two PL subspaces. Then $A$ and $B$ are said to be in \textbf{general position in the stratification } of $X$ if
\begin{align*}
dim(A \cap B \cap \mathcal{X}_k) \leq dim(A \cap \mathcal{X}_k) + dim(B \cap \mathcal{X}_k) -k
\end{align*}
for all $k$, i.e. if $A \cap \mathcal{X}_k$ and $B \cap \mathcal{X}_k$ are in general position in $\mathcal{X}_k$, for each $k$.
\end{definition}

\begin{theorem}\label{mccroryisotopy}
Let $A,B,C \subset X$ be closed PL subspaces with $C \subset B$. Given $\epsilon > 0$ there exists a stratum-preserving PL isotopy $H: X \times I \to X$ (i.e. for any $t \in I$ we have $H(X_k \times \{ t \} ) \subset X_k$, for all $k$) with the following properties:
\begin{itemize}
\item[1.] $|H(x,t)-x| < \epsilon$ for all $x$ and $t$,
\item[2.] $H(x,t)=x$ for all $x \in C$ and all $t$,
\item[3.] $H(x,0)=x$ for all $x \in X$,
\item[4.] $H((B-C) \times \{ 1 \})$ and $A$ are in general position in the stratification of $X$.
\end{itemize}
In other words, $B-C$ can be moved into general position with respect to $A$ by an arbitrarily small isotopy, keeping $A$ and $C$ untouched.
\end{theorem}

\begin{proof}
See \cite{mccrory}. In \cite{zeeman}, ch.6, Zeeman introduces a technique which he calls "local shifts" to move a subpolyhedron $B$ of a PL manifold $M$ into general position with respect to some other subpolyhedron $A$ of $M$. Roughly speaking, a local shift of a simplex $\sigma \subset B$ is an arbitrarily small move (isotopy) of the closed star of $\sigma$ in $M$, in such a way that the boundary of $\sigma$ remains unaltered. The latter property guarantees that these local shifts extend to a global isotopy, which is arbitrarily small and keeps $C$ and $X-(B-C)$ fixed. McCrory picks up this idea and shows in a finite induction on the strata, that the isotopies on the pure strata - in terms of local shifts - give rise to an isotopy $H$ on $X$ with the desired properties. Moreover, the local definition of $H$ ensures that 
\begin{align*}
dim(H(\sigma \times I)) \leq dim(\sigma)+1,
\end{align*}
for each simplex $\sigma \subset X$.
\end{proof}

\section{Theories of $\overline{p}$-allowable bordism classes}
 
\begin{definition}
For a compact PL space $P$, a PL map $f: P \to X$ is called \textbf{($\boldsymbol{\overline{p}}$,i)-allowable} if $dim(f(P)) \leq i$ and 
\begin{align*}
dim(f(P) \cap X_{n-k}) \leq i-k+ \overline{p}(k),
\end{align*}
for all $k \geq 2$.
\end{definition}

\begin{remark}
Note that any representative of the isomorphism class (in the sense of Def.\ref{isomorphiedef}) of a $(\overline{p},i)$-allowable map is $(\overline{p},i)$-allowable as well. Indeed, consider a commutative diagram
\begin{xy}
  \xymatrix{
      P \ar[rr]^h \ar[rd]_f  &     &  Q \ar[dl]^g  \\
                             &  X &
  }
\end{xy}
with a PL isomorphism $h$ and assume $f$ to be $(\overline{p},i)$-allowable. We then have
\begin{align*}
dim(g(Q) \cap X_{n-k}) = dim(f \circ h^{-1} (Q) \cap X_{n-k})=dim(f(P) \cap X_{n-k}) \leq i-k+ \overline{p}(k).
\end{align*}
\end{remark}

\begin{definition}
Let $\mathcal{L}$ be a theory with singularities (as in ch.\ref{bordismhomology}). Given two $\mathcal{L}_i$-manifolds $P,Q$ and PL maps $f: P \to X$ and $g: Q \to X$, we say that $f$ and $g$ are \textbf{oriented ($\boldsymbol{\overline{p}}$,i+1)-bordant} if $f$ and $g$ are bordant (in the sense of Def.$\ref{bordismusdef}$) via an oriented $({\overline{p}}$,i+1)-allowable PL bordism. In this situation, we write $f \sim_{i+1}^{\overline{p}} g$.
\end{definition}

\begin{definition}
If $\mathcal{L}$ is a theory with singularities, let $Isom_i^{\mathcal{L}, \overline{p}}(X)$ denote the set of isomorphism classes of PL maps $f: P \to X$ which are oriented $(\overline{p},i+1)$-bordant to some $(\overline{p},i)$-allowable PL map, where $P$ varies over all closed, compact, oriented $\mathcal{L}_i$- manifolds.
\end{definition}

\begin{lemma}\label{producttriangulation}
If $K$ is a simplicial complex, then there exists a triangulation of $|K| \times I$ whose set of vertices consists of precisely two vertices $(v,0), (v,1)$ for each vertex $v$ in $K$.
\end{lemma}

\begin{proof}
Let $\sigma=[v_0,...,v_k]$ be a simplex in $K$. Then $\sigma \times I$ has $A_0=(v_0,0),...,A_k=(v_k,0)$ and $B_0=(v_0,1),...,B_k=(v_k,1)$ as vertices. For each $0 \leq i \leq k$, consider simplices of the form $[A_0,...,A_i,B_i,...,B_k]$. Then the complex generated by those simplices provides a triangulation of $| \sigma| \times I$. Proceedig this way for every $\sigma \in K$, we obtain a triangulation of $|K| \times I$ with the desired properties.
\end{proof}
\begin{prop}\label{intersectioneqrelation}
$\sim_{i+1}^{\overline{p}}$ is an equivalence relation on $Isom_i^{\mathcal{L}, \overline{p}}(X)$.
\end{prop}

\begin{proof}
Let $[f] \in Isom_i^{\mathcal{L}, \overline{p}}(X)$. Then $f$ is $(\overline{p},i+1)$-bordant to some $(\overline{p},i)$-allowable map via a bordism $F$, say. Since $F_{|dom(f)}=f$, we find that
\begin{align*}
dim(im(f) \cap X_{n-k}) = dim(im(F_{|dom(f)}) \cap X_{n-k}) &\leq dim(im(F) \cap X_{n-k}) \\ &\leq i+1-k+ \overline{p}(k).
\end{align*}
Hence, $f$ is $(\overline{p},i+1)$-allowable and we can define a $(\overline{p},i+1)$-allowable PL map \begin{align*}
G: dom(f) \times I \to X
\end{align*} as follows. First since $f$ is PL, we can triangulate $dom(f)$ and $X$ in such a way that $f$ is simplicial with respect to these triangulations. Considering this triangulation of $dom(f)$, we triangulate $dom(f) \times I$ like in Lemma $\ref{producttriangulation}$. Then, for each vertex $v$ of $dom(f)$, set $G(v,0)=G(v,1)=f(v)$ and extend linearly over the simplices of $dom(f) \times I$. We have $\partial(dom(f) \times I)= dom(f) \times \{ 0 \} \sqcup -dom(f) \times \{  1 \}$, and clearly $G_{|dom(f) \times \{  0 \} }=G_{|dom(f) \times \{  1 \} } = f$. This proves reflexivity of the relation. For symmetry, note once more that disjoint union commutes up to PL isomorphism. To see transitivity, let $f \sim_{i+1}^{\overline{p}} g$ and $g \sim_{i+1}^{\overline{p}} h$ via bordisms $F$ and $G$, respectively. Let
\begin{align*}
H: dom(F) \cup_{dom(g)} dom(G) \to X
\end{align*}
be the unique map with $H_{|dom(F)}=F$ and $H_{|dom(G)}=G$, obtained by glueing along $dom(g)$. By assumption, $F$ and $G$ are PL, and so is $H$. Moreover, $H$ is $(\overline{p},i+1)$-allowable, since
\begin{align*}
dim(im(H) \cap X_{n-k}) &= dim((im(F) \cup im(G)) \cap X_{n-k})\\ & = dim(im(F) \cap X_{n-k} \cup im(G) \cap X_{n-k}) \\ &= max(dim(im(F) \cap X_{n-k}), dim(im(G) \cap X_{n-k})) \\ &\leq i+1-k+ \overline{p}(k),
\end{align*}
where the last inequality holds as $F$ and $G$ are both $(\overline{p},i+1)$-allowable. We conclude $f \sim_{i+1}^{\overline{p}} h$ via $H$, which completes the proof.
\end{proof}

\begin{definition}
For a theory with singularities $\mathcal{L}$, we define the corresponding $i$-th $\overline{p}$\textbf{-allowable bordism set} by 
\begin{align*}
I \Omega_i^{\mathcal{L}, \overline{p}}(X) := Isom_i^{\mathcal{L}, \overline{p}}(X)/ \sim_{i+1}^{\overline{p}} .
\end{align*}
\end{definition}

\begin{prop}
$I \Omega_i^{\mathcal{L}, \overline{p}}(X) $ is an abelian group with respect to disjoint union.
\end{prop}

\begin{proof}
Let $[f],[g] \in I \Omega_i^{\mathcal{L}, \overline{p}}(X) $ with $f$ and $g$ both $(\overline{p},i)$-allowable. Then $f \sqcup g$ is $(\overline{p},i)$-allowable for the same reason as in the proof of transitivity in Prop.$\ref{intersectioneqrelation}$. This argument also shows that $\sqcup$ is a well-defined operation in $I \Omega_i^{\mathcal{L}, \overline{p}}(X)$: Let $f' \in [f],g' \in [g]$ be representatives. Then there are PL bordisms $F$ between $f$ and $f'$, $G$ between $g$ and $g'$, both $(\overline{p},i+1)$-allowable. This leads to a $(\overline{p},i+1)$-allowable PL bordism $F \sqcup G$ between $f \sqcup g$ and $f' \sqcup g'$, and so it makes sense to define $[f] \sqcup [g]:=[f \sqcup g]$. As usual, the zero element is given by $[\emptyset \to X]$ and the inverse of $[f: P \to X]$ is $[f: -P \to X]$. Associativity and commutativity clearly hold.
\end{proof}

\begin{prop}\label{Iomegazero}
$I \Omega_0^{\mathcal{L}, \overline{p}}(X)= \Omega_0^{\mathcal{L}}(X-X_{n-2})$ for any theory with singularities $\mathcal{L}$, independently of the perversity $\overline{p}$.
\end{prop}

\begin{proof}
A map $f:P \to X$ is $(\overline{p},0)$-allowable if and only if $dim(f(P)) \leq 0$ and
\begin{align*}
 dim(f(P) \cap X_{n-k}) \leq -k+\overline{p}(k) \leq -k+ \overline{t}(k) =-2
\end{align*} for all $k \geq 2$. This means that any $(\overline{p},0)$-allowable map necessarily misses the singular set $X_{n-2}$, and similarly any $(\overline{p},1)$-allowable bordism misses $X_{n-2}$. Conversely, each class in $ \Omega_0^{\mathcal{L}}(X-X_{n-2})$ defines a class in $I \Omega_0^{\mathcal{L}, \overline{p}}(X)$, as the admissibility conditions are empty conditions for such classes.
\end{proof}

\begin{remark}
Note that we did not use the specific structure of $\mathcal{L}$ to define $\overline{p}$-allowable bordism groups, i.e. whenever it makes sense to talk about a bordism relation on a class of compact polyhedra (e.g. as in $\cite{stong}$), we can assign a corresponding $\overline{p}$-allowable bordism theory to it in an analogous manner as above. 
\end{remark}

Now, let $\mathcal{L}$ be a theory with singularities and let $T$ be a triangulation of $X$, subordinate to the stratification. Define the basic sets $Q_i^{\overline{p}}$ with respect to $T$, as in chapter $\ref{basicsetschapter}$. Since $\Omega_i^{\mathcal{L}}$ is a functor for each $i$, the inclusions $Q_i^{\overline{p}} \subset Q_{i+1}^{\overline{p}}$ induce maps
\begin{align*}
\Omega_i^{\mathcal{L}}(Q_{i}^{\overline{p}}) \to \Omega_i^{\mathcal{L}}(Q_{i+1}^{\overline{p}}).
\end{align*}
An element in $Im(\Omega_i^{\mathcal{L}}(Q_{i}^{\overline{p}}) \to \Omega_i^{\mathcal{L}}(Q_{i+1}^{\overline{p}}))$ is represented by a map $f: P \to Q_{i+1}^{\overline{p}}$, where $P$ is a closed $\mathcal{L}_i$-manifold and such that there exist a closed $\mathcal{L}_i$-manifold $R$, an $\mathcal{L}_{i+1}$-manifold $W$ with $\partial W \cong P \sqcup -R$, and a bordism $F: W \to Q_{i+1}^{\overline{p}}$ with $F_{|P}=f$ and $g:=F_{|R}: R \to Q_{i}^{\overline{p}}$. We may assume $g$ to be PL, otherwise the simplicial approximation theorem guarantees the existence of a homotopy (and thus a bordism) over $Q_i^{\overline{p}}$ which moves $g$ into a PL map. Moreover, if $g'$ is another PL map which is bordant to $g$, then the corresponding bordism may be assumed to be PL, again by simplicial approximation. By Prop.$\ref{basicsetsallowable}$, $g$ is $(\overline{p},i)$-allowable and the bordism between $g$ and $g'$ is $(\overline{p},i+1)$-allowable. Thus, we have a well-defined homomorphism
\begin{align*}
\Phi : Im(\Omega_i^{\mathcal{L}}(Q_{i}^{\overline{p}}) \to \Omega_i^{\mathcal{L}}(Q_{i+1}^{\overline{p}})) \to I\Omega_i^{\mathcal{L}, \overline{p}}(X)
\end{align*}
by mapping $[g]$ to the corresponding class in $I\Omega_i^{\mathcal{L}, \overline{p}}(X)$ that is represented by $g$.

\begin{theorem}\label{maintheorem}
In the above setting, the map
\begin{align*}
\Phi : Im(\Omega_i^{\mathcal{L}}(Q_{i}^{\overline{p}}) \to \Omega_i^{\mathcal{L}}(Q_{i+1}^{\overline{p}})) \to I\Omega_i^{\mathcal{L}, \overline{p}}(X)
\end{align*}
is an isomorphism for $i \geq 1$.
\end{theorem}

\begin{proof}
First we show surjectivity. Let $[f: P \to X] \in I\Omega_i^{\mathcal{L}, \overline{p}}(X)$. We know that $f(P)$ is a compact polyhedron since it is the image of a compact polyhedron under a PL map. Now let $\overline{q}$ be the complementary perversity to $\overline{p}$ and let
\begin{align*}
H: X \times I \to X
\end{align*}
be an isotopy which moves $f(P)$ into general position with respect to $Q_{n-i+1}^{\overline{q}} \cap |T_{(n-2)}|$ in the stratification of $X$, as in Theorem $\ref{mccroryisotopy}$. (Recall that $|T_{(n-2)}|$ denotes the $(n-2)$-skeleton of $T$.) We define a bordism 
\begin{align*}
F: P \times I \to X
\end{align*}
by setting $F(p,s):= H(f(p),s)$. By construction of $H$, we have $F(-,0)=f$ and since $H$ is PL, so is $F$. Moreover, the construction of $H$ ensures that 
\begin{align*}
H(\sigma \times I) \cap X_{n-k} = H(\sigma \cap X_{n-k} \times I)
\end{align*}
for any simplex $\sigma \subset f(P)$ and any stratum $X_{n-k}$, and if $\tau \subset f(P) \cap X_{n-k}$ is some simplex, we have 
\begin{align*}
dim(H(\tau \times I)) \leq dim(\tau) +1 \leq i-k+ \overline{p}(k)+1=i+1-k + \overline{p}(k),
\end{align*}
where the second inequality holds since $f$ is $(\overline{p},i)$-allowable. Putting these facts together, we see that $F$ is a $(\overline{p},i+1)$-allowable bordism and a similar argument shows that \\ $g:=F(-,1)$ is $(\overline{p},i)$-allowable. Since the image of $g$ is in general position with respect to $Q_{n-i+1}^{\overline{q}} \cap |T_{(n-2)}|$, we have
\begin{align*}
&dim(g(P) \cap Q_{n-i+1}^{\overline{q}} \cap |T_{(n-2)}| \cap \mathcal{X}_{n-k}) \\ \leq \ &dim(g(P) \cap \mathcal{X}_{n-k}) + dim(Q_{n-i+1}^{\overline{q}} \cap |T_{(n-2)}| \cap \mathcal{X}_{n-k}) - (n-k) \\ \leq \ &i-k+ \overline{p}(k)+n-i+1-k+ \overline{q}(k)-(n-k) \\ =\  &-k+1+ \overline{t}(k) =-1
\end{align*}
whenever $k \geq 2$. Note that the second inequality holds since $g$ is $(\overline{p},i)$-allowable and since $Q_{n-i+1}^{\overline{q}}$ is $(\overline{q},n-i+1)$-allowable, by Prop.\ref{basicsetsallowable}. Furthermore, we observe
\begin{align*}
&dim(g(P) \cap Q_{n-i+1}^{\overline{q}} \cap |T_{(n-2)}| \cap \mathcal{X}_{n}) \\ \leq \ &dim(g(P) \cap \mathcal{X}_{n}) + dim(Q_{n-i+1}^{\overline{q}} \cap |T_{(n-2)}| \cap \mathcal{X}_{n}) - n \\ \leq \ &i+n-i-1-n = -1,
\end{align*}
where the second inequality uses the fact that $dim(Q_{n-i+1}^{\overline{q}} \cap |T_{(n-2)}|) = n-i-1$, which is known from the proof of Prop.\ref{retractions}. Summarizing, we see that the intersection of $g(P) \cap Q_{n-i+1}^{\overline{q}} \cap |T_{(n-2)}|$ with each pure stratum is empty, and so
\begin{align*}
g(P) \cap Q_{n-i+1}^{\overline{q}} \cap |T_{(n-2)}| = \emptyset.
\end{align*}
In other words, $g$ is effectively a map 
\begin{align*}
g: P \to X-(Q_{n-i+1}^{\overline{q}} \cap |T_{(n-2)}|).
\end{align*}
We would now like to use the deformation retraction $r:  Q_{n-i+1}^{\overline{q}} \cap |T_{(n-2)}| \to Q_i^{\overline{p}}$ from Prop.\ref{retractions} to further homotope $g$. The problem is that $r$ is not PL. To circumvent this issue, we choose a triangulation of $P$ so fine that the image of each simplex under $g$ is contained in some simplex of $T^1$. With respect to this triangulation of $P$, we now triangulate $P \times I$ as in Lemma \ref{producttriangulation}. Then the vertices of $P \times I$ are contained in $P \times \{ 0 \} \cup P \times \{ 1 \}$ and for each vertex $v \in P$, we define $G(v,0):=g(v)$ and $G(v,1):= r \circ g (v)$. Extending linearly over the simplices of $P \times I$ yields a bordism 
\begin{align*}
G: P \times I \to X
\end{align*}
between $g$ and some map $h:=G(-,1):P \to Q_i^{\overline{p}}$, where $G$ can be shown to be $(\overline{p},i+1)$-allowable, see \cite{GM}, p.148. It now follows that 
$\Phi([h])=[f].$ \\
To see injectivity of $\Phi$, suppose we are given a PL map $f: P \to Q_i^{\overline{p}}$ with $\Phi([f])=0$. This means that there exists a $(\overline{p},i+1)$-allowable bordism $F:W \to X$, where $W$ is an $\mathcal{L}_{i+1}$-manifold with $\partial W \cong P$ and $F_{|P}=f$. Let
\begin{align*}
H: X \times I \to X
\end{align*}
be an isotopy which moves $F(W)-(F(W) \cap Q_i^{\overline{p}})$ into general position with respect to $Q_{n-i}^{\overline{q}} \cap |T_{(n-2)}|$ in the stratification of $X$ and such that $F(W) \cap Q_i^{\overline{p}}$ remains fixed under $H$, as in Theorem \ref{mccroryisotopy}. We define a map
\begin{align*}
\mathcal{G}: W \times I \to X
\end{align*}
by setting $\mathcal{G}(w,s):=H(F(w),s)$. As in the proof of surjectivity, we see that $\mathcal{G}(-,1)$ is in fact a map
\begin{align*}
G:=\mathcal{G}(-,1): W \to X-(Q_{n-i}^{\overline{q}} \cap |T_{(n-2)}|)
\end{align*}
and that $G$ is $(\overline{p},i+1)$-allowable. Moreover, for any $p \in P$ we have
\begin{align}
G(p)=H(F(p),1)=H(f(p),1)=f(p),
\end{align}
since $H$ keeps $F(W) \cap Q_i^{\overline{p}}$ fixed. We conclude that $G$ is a $(\overline{p},i+1)$-allowable null-bordism for $f$. Composition of $G$ with the retraction
\begin{align*}
r: X-(Q_{n-i}^{\overline{q}} \cap |T_{(n-2)}|) \to Q_{i+1}^{\overline{p}}
\end{align*}
leads to a map
$r \circ G : W \to Q_{i+1}^{\overline{p}}$, which is a null-bordism for $f$ over $Q_{i+1}^{\overline{p}}$ by $(4.1)$ and since $r$ is the identity on $f(P) \subset Q_{i+1}^{\overline{p}}$, as it is a retraction. Consequently, $[f] =0$ in $Im(\Omega_i^{\mathcal{L}}(Q_{i}^{\overline{p}}) \to \Omega_i^{\mathcal{L}}(Q_{i+1}^{\overline{p}}))$, and injectivity is proven.
\end{proof}

We proceed by giving some corollaries of the previous theorem.

\begin{corollar}
If $I\Omega_*^{\overline{p},ord}$ denotes the $\overline{p}$-allowable bordism theory corresponding to the theory of singularities that defines $\Omega_*^{ord}$, we have
\begin{align*}
I\Omega_i^{\overline{p},ord}(X) \cong IH_i^{\overline{p}}(X)
\end{align*}
for any PL stratified pseudomanifold $X^n$ and any $i \geq 0$.
\end{corollar}

\begin{proof}
For $i \geq 1$, we have 
\begin{align*}
 I\Omega_i^{\overline{p},ord}(X) \cong Im(\Omega_i^{ord}(Q_{i}^{\overline{p}}) \to \Omega_i^{ord}(Q_{i+1}^{\overline{p}}))
\end{align*}
by Theorem $\ref{maintheorem}$. Furthermore, the uniqueness theorem for ordinary homology ensures that
\begin{align*}
Im(\Omega_i^{ord}(Q_{i}^{\overline{p}}) \to \Omega_i^{ord}(Q_{i+1}^{\overline{p}})) \cong Im(H_i(Q_{i}^{\overline{p}}) \to H_i(Q_{i+1}^{\overline{p}}))
\end{align*}
since $\Omega_*^{ord}$ and $H_*$ are naturally isomorphic. Using Theorem $\ref{theorembasicsets}$, we obtain
\begin{align*}
Im(H_i(Q_{i}^{\overline{p}}) \to H_i(Q_{i+1}^{\overline{p}})) \cong IH_i^{\overline{p}}(X),
\end{align*}
which proves the claim in the case if $i \geq 1$. For $i=0$, we have
\begin{align*}
I\Omega_0^{\overline{p},ord}(X)=\Omega_0^{ord}(X-X_{n-2}) \cong H_0(X-X_{n-2}),
\end{align*}
by Prop $\ref{Iomegazero}$ and a similar observation in terms of PL chains shows 
\begin{align*}
 H_0(X-X_{n-2})=IH_0^{\overline{p}}(X),
\end{align*}
see \cite{GM}, p.139.
\end{proof}

\begin{corollar}\label{IOmegaindependentstrat}
$I\Omega_i^{\mathcal{L}, \overline{p}}(X)$ is independent of the stratification of $X$, for each $i \geq 1$.
\end{corollar}

\begin{corollar}
If $T$ is an arbitrary triangulation of $X$ and if the $Q_i^{\overline{p}}$ are defined with respect to $T$, then
\begin{align*}
I\Omega_i^{\mathcal{L}, \overline{p}}(X)\cong Im(\Omega_i^{\mathcal{L}}(Q_{i}^{\overline{p}}) \to \Omega_i^{\mathcal{L}}(Q_{i+1}^{\overline{p}})),
\end{align*}
for each $i \geq 1$.
\end{corollar}

The proofs of the previous two Corollaries are precisely the proofs of Corollaries \ref{IHindependentstrat} and \ref{IHindependenttriang}, respectively with $IH_*^{\overline{p}}$ replaced by $I\Omega_*^{\mathcal{L}, \overline{p}}$ and $H_*$ replaced by $\Omega_*^{\mathcal{L}}$.

\chapter{Basic properties of $\overline{p}$-allowable bordism theories}

This short chapter collects some of the basic facts about $\overline{p}$-allowable bordism theories. Rather than studying certain theories for special classes of singularities, we tried to focus on the behavior of these theories in a possibly general setting.
Throughout this chapter, let $X$ be a PL stratified pseudomanifold of dimension $n$.
\section{Induced maps}

In order to compare different $\overline{p}$-allowable bordism groups, we would like to have induced maps for varying parameters. There are at least three such parameters that might be varied, namely the perversity function, the base pseudomanifold and the theory of singularities itself. In the following we explore certain situations in which there exist such induced maps. The proofs are essentially observations.

\begin{prop}
Let $\mathcal{L}$ be a fixed theory with singularities and assume we are given two perversities $\overline{p}, \overline{q}$ with $\overline{p} \leq \overline{q}$. We then have an induced homomorphism
\begin{align*}
I \Omega_i^{\mathcal{L}, \overline{p}}(X) \to I \Omega_i^{\mathcal{L}, \overline{q}}(X),
\end{align*}
for any $i \geq 0$.
\end{prop}

\begin{proof}
By assumption, we have $\overline{p} \leq \overline{q}$ and therefore any $(\overline{p},i)$-allowable map is $(\overline{q},i)$-allowable as well. Similarly, any $(\overline{p},i+1)$-allowable bordism is $(\overline{q},i+1)$-allowable, and so any class in $I \Omega_i^{\mathcal{L}, \overline{p}}(X)$ defines a class in $I \Omega_i^{\mathcal{L}, \overline{q}}(X)$.
\end{proof}

\begin{prop}
Assume we are given a perversity $\overline{p}$ and two theories with singularities $\mathcal{L}$ and $\mathcal{L'}$, which satisfy $\mathcal{L}_n \subset \mathcal{L}_n'$ for each $n$. We then have an induced homomorphism 
\begin{align*}
I \Omega_i^{\mathcal{L}, \overline{p}}(X) \to I \Omega_i^{\mathcal{L'}, \overline{p}}(X),
\end{align*}
for each $i \geq 0$.
\end{prop}

\begin{proof}
Let $[f: P \to X] \in I \Omega_i^{\mathcal{L}, \overline{p}}(X)$. Then $P$ is a closed $\mathcal{L}_i$-manifold and by assumption it is a closed $\mathcal{L}_i'$-manifold, too. Similarly, any $\mathcal{L}$-bordism is an $\mathcal{L'}$-bordism as well and since the admissibility contraints are the same in either of the two groups, $[f]$ defines a class in $I \Omega_i^{\mathcal{L'}, \overline{p}}(X)$.
\end{proof}

\begin{prop}
For any theory with singularities $\mathcal{L}$, any perversity $\overline{p}$, and for all $i \geq 0$ there exists a homomorphism
\begin{align*}
I \Omega_i^{\mathcal{L}, \overline{p}}(X) \to \Omega_i^{\mathcal{L}}(X)
\end{align*}
by discarding admissibility conditions that are imposed on maps and bordisms. If $X=M$ is a PL manifold, these maps are isomorphisms.
\end{prop}

\begin{proof}
It is clear that such a homomorphism exists for each $i \geq 0$, since each $\overline{p}$-allowable $\mathcal{L}$-bordism class particularly defines an $\mathcal{L}$-bordism class of the same degree. If $X=M$ is a PL manifold, note that there is a trivial stratification of $M$, given by $M \supset \emptyset$, and Cor.\ref{IOmegaindependentstrat} tells us that we can define $I \Omega_i^{\mathcal{L}, \overline{p}}(M)$ with respect to this stratification. But then the admissibility conditions for maps and bordisms reduce to empty conditions, as there do not exist strata of codimension at least two. This means that there is a one-two-one correspondance between $\overline{p}$-allowable $\mathcal{L}$-bordism classes and $\mathcal{L}$-bordism classes over $M$ and thus the induced maps
\begin{align*}
I \Omega_i^{\mathcal{L}, \overline{p}}(M) \to \Omega_i^{\mathcal{L}}(M)
\end{align*}
are isomorphisms for each $i \geq 0$.
\end{proof}

\section{Relative groups}

Let $U \subset X$ be an open subset. Then $U$ admits a canonical PL structure, which is induced by the one of $X$: $S$ is an admissible triangulation if there exists an admissible triangulation $T$ of $X$ and a subdivision $S' \lhd S$ such that each simplex of $S'$ is contained in some simplex of $T$. Moreover, $U$ admits a stratification
\begin{align*}
U \supset U \cap X_{n-2} \supset U \cap X_{n-3} \supset \cdots \supset U \cap X_{0} \supset \emptyset,
\end{align*}
induced by the stratification of $X$. In this way, $U$ is a PL stratified pseudomanifold in its own right, and so it makes sense to talk about $I\Omega_*^{\mathcal{L},\overline{p}}(U)$. For each $i \geq 0$, there is an induced homomorphism
\begin{align*}
I\Omega_i^{\mathcal{L},\overline{p}}(U) \to I\Omega_i^{\mathcal{L},\overline{p}}(X),
\end{align*}
since if $[f: P \to U] \in I\Omega_i^{\mathcal{L},\overline{p}}(U)$ with $f$ being $(\overline{p},i)$-allowable, then 
\begin{align*}
incl \circ f: P \to U \xhookrightarrow{} X
\end{align*}
is $(\overline{p},i)$-allowable over $X$ as we have
\begin{align*}
dim(incl \circ f(P) \cap X_{n-k})&=dim(f(P) \cap X_{n-k} \cap U)=dim(f(P) \cap U_{n-k}) \\
&\leq i-k+ \overline{p}(k),
\end{align*}
and something similar holds for bordisms. Therefore, $[incl \circ f: P \to U \xhookrightarrow{} X]$ defines a class in  $I\Omega_i^{\mathcal{L},\overline{p}}(X)$.

\begin{definition}
With the above notation and for a pair of compact PL spaces $(P, Q)$, a PL map $f: (P, Q) \to (X,U)$ is called $(\overline{p},i)$-allowable if $dim(f(P)) \leq i$,  $dim(f(Q)) \leq i-1$ and 
\begin{align*}
&dim(f(P) \cap X_{n-k}) \leq i-k+ \overline{p}(k), \\
&dim(f(Q) \cap U_{n-k}) \leq i-1-k+ \overline{p}(k),
\end{align*}
for all $k \geq 2$.
\end{definition}

\begin{definition}
In the above setting, we define the $i$-th \textbf{relative $\overline{p}$-allowable bordism set} $I\Omega_i^{\mathcal{L},\overline{p}}(X,U)$ to be the set of $(\overline{p},i)$-allowable PL maps $f: (P,\partial P) \to (X,U)$ modulo $(\overline{p},i+1)$-allowable bordisms (in the sense of Def.\ref{bordismusdef}) $F:(W,Z) \to (X,U)$, where $P$ ranges over all compact, oriented $\mathcal{L}_i$-manifolds with boundary and $(W,Z)$ ranges over all pairs of compact, oriented $\mathcal{L}_{i+1}$-manifolds with boundary.
\end{definition}

\begin{remark}
In order to see that the above Definition does not lead to set-theoretic problems, one can define an appropriate equivalence relation on the set of isomorphism classes of PL maps $(P, \partial P) \to (X,U)$, similarly as in Prop.\ref{intersectioneqrelation}. It is easily seen that relative $\overline{p}$-allowable bordism sets are in fact abelian groups with respect to disjoint union.
\end{remark}
We then have a homomorphism
\begin{align*}
I\Omega_i^{\mathcal{L},\overline{p}}(X) \to I\Omega_i^{\mathcal{L},\overline{p}}(X,U)
\end{align*}
by mapping a class $[P \to X] \in I\Omega_i^{\mathcal{L},\overline{p}}(X)$ to $[(P,\emptyset) \to (X,U)] \in I\Omega_i^{\mathcal{L},\overline{p}}(X,U)$ and this leads to the following 
\begin{prop}
In the above setting, we have the long exact sequence
\begin{equation*}
\begin{xy}
\xymatrix{ 
\cdots \ar[r]^{}     &   I\Omega_{i+1}^{\mathcal{L},\overline{p}}(X,U) \ar[r]^{\ \ \partial_{i+1}}    &   I\Omega_i^{\mathcal{L},\overline{p}}(U) \ar[r]^{}   &  I\Omega_i^{\mathcal{L},\overline{p}}(X) \\
  \ar[r]^{}   &   I\Omega_i^{\mathcal{L},\overline{p}}(X,U)  \ar[r]^{\partial_i}   &  I\Omega_{i-1}^{\mathcal{L},\overline{p}}(U) \ar[r]^{}   &   \cdots
}
\end{xy}
\end{equation*}
for all $i \geq 1$, where $\partial$ is induced by restriction to the boundary.
\end{prop}

\begin{proof}
This is essentially the proof of Theorem $\ref{Omegageneralizedhomology}$ together with paying careful attention to admissibility constraints.
\end{proof}

\begin{thebibliography}[
\bibitem{pltopo}C.P. Rourke, B.J. Sanderson, \textit{Introduction to piecewise-linear topology}, Ergebnisse der Mathematik und ihrer Grenzgebiete, Band 69, Springer-Verlag, Heidelberg and New York, 1978.
\bibitem{hatcher}A. Hatcher, \textit{Algebraic Topology}, Cambridge University Press, 2002, p.120.
\bibitem{lecturenotes}I. M. Singer and J. A. Thorpe, \textit{Lecture Notes on Elementary Topology and Geometry}, M.I.T. and Haverford College.
\bibitem{GM}M. Goresky, R. McPherson, \textit{Intersection Homology Theory}, Topology Vol. 19, 1980, p. 135-162.
\bibitem{BRS}S. Buoncristiano, C. Rourke and B. Sanderson, \textit{A Geometric Approach to Homology Theory}, London Math. Soc. Lecture notes No. 18. Cambridge University Press (1976).
\bibitem{stong}R. E. Stong, \textit{Notes on cobordism theory}, Princeton University Press, Princeton, 1968.
\bibitem{mccrory}C. McCrory, \textit{Stratified general position}, Algebraic and Geometric Topology, p.142-146. Springer Lecture Notes in Mathematics, No. 644. Springer-Verlag, New York (1978).
\bibitem{zeeman}E. C. Zeeman, \textit{Seminar on Combinatorial Topology}, I.H.E.S. Paris and the University of Warwick at Conventry, 1963-1966.
\bibitem{hudson}J.F.P. Hudson, \textit{Piecewise linear topology}, University of Durham, 1969.
\bibitem{banagl}M. Banagl, \textit{Topological Invariants of Stratified Spaces}, Springer Monographs in Mathematics, Springer-Verlag Berlin Heidelberg, 2007.
\bibitem{kreck}M. Kreck, \textit{Differential Algebraic Topology,} American Mathematical Society Providence, Rhode Island, 2010.
\bibitem{bredon}G. Bredon, \textit{Topology and Geometry,} Springer-Verlag New York and Heidelberg, 1993.
\bibitem{dold}A. Dold, \textit{Lectures on Algebraic Topology,} Springer-Verlag Berlin, Heidelberg, New York, 1980.
\bibitem{borel}A. Borel et al. \textit{Intersection Cohomology,} Birkhaeuser, Boston, Basel, Stuttgart, 1984.
\bibitem{ES}S. Eilenberg, N. Steenrod, \textit{Foundations of Algebraic Topology,} Princeton University Press, 1952.
\bibitem{kirwan}F. Kirwan, \textit{An introduction to intersection homology theory,} University of Oxford, 1988.
\bibitem{adams}J.F. Adams, \textit{Stable homotopy and generalised homology,} Chicago Lectures in Mathematics, 1974.
\bibitem{friedman}G. Friedman, \textit{Stratified and unstratified bordism of pseudomanifolds,} 	Texas Christian University, 2015.
\bibitem{rourkestratifications}C. Rourke, B. Sanderson, \textit{Homology stratifications and intersection homology,} Geometry \& Topology Monographs Volume 2: Proceedings of the Kirbyfest, p.455-472.
\bibitem{comezana}G.R. Comezana, \textit{Bordism of layered cycles and generalized intersection homology theory,} unpublished Ph.D. thesis, Graduate School-New Brunswick, 1991.
\end{thebibliography}

\end{document}
