\documentclass[11pt]{book}
\usepackage[english]{babel}
\usepackage[all]{xy}
\usepackage{hyperref}
\usepackage{amsthm}
\usepackage{amsmath}
\usepackage{amssymb}

\newtheorem{prop}{Proposition}
\newtheorem{lemma}{Lemma}
\newtheorem{theorem}{Theorem}
\newtheorem{definition}{Definition}
\newtheorem{remark}{Remark}



\begin{document}

\tableofcontents

\chapter{Some PL topology}
\label{sec:Einleitung}

In this chapter we will collect some basic facts from the piecewise-linear category. Most of the proofs are omitted and can be found in standard references like \cite{pltopo}. If the reader is already familiar with basics from PL topology, the chapter may be skipped without loss of continuity.


\begin{definition}
Let $v_0,...,v_k \in \mathbb{R}^n$ be points in some affine space such that $\{ v_1-v_0,...,v_k-v_0 \}$ is a set of linearly independent vectors. We call

\begin{equation}
[v_0,...,v_k] := \Biggl \{  \lambda_0v_0+...+ \lambda_kv_k \Bigg |  \sum_{i=0}^k \lambda_i =1 \text{ and } \lambda_i \geq 0 \text{ for all i} \Biggr \}
\end{equation}
the simplex spanned by $\{v_0,...,v_k \}$. Its dimension is $k$ and we call it a \textbf{$k$-simplex} for short. The points that span a simplex are called vertices. For a simplex $\sigma$ we say that $\tau$ is a \textbf{face} of $\sigma$ if $\tau$ is a simplex spanned by a nonempty subset of the vertices of $\sigma$ and we abbreviate this by writing $\tau < \sigma$.
\end{definition}

\begin{definition}
A \textbf{simplicial complex} $K$ is a set of simplices that satisfies the following conditions:

\begin{itemize}
\item Every face of a simplex in $K$ is also contained in $K$.
\item The intersection of any two simplices $\sigma, \tau \in K$ is either empty or a face of both $\sigma$ and $\tau$.
\end{itemize}

We define the \textbf{geometric realization} of $K$ by $|K|:= \bigcup \{ \sigma | \sigma \in K \}$.

\end{definition}


\begin{definition}
Suppose that there are two simplicial complexes $K$ and $K'$ such that $|K| = |K'|$. If every simplex of $K'$ is contained in some simplex of $K$, we say that $K'$ is a \textbf{subdivision} of $K$ and write $K' \lhd K$.
\end{definition}


\begin{definition}
A topological space $X$ is said to be $\mathbf{triangulable}$ if there exists a simplicial complex $T$ and a homeomorphism $\phi : |T| \to X$. The triple $(T,X, \phi)$ is called a \textbf{triangulation} of $X$. In this situation we will simply say that $T$ is a triangulation of $X$, by abuse of notation.
\end{definition}


\begin{definition}
A \textbf{PL space} is a pair $(X,\mathcal{T})$ consisting of a topological space $X$ and a class $\mathcal{T}$ of locally finite triangulations of $X$ which satisfies the following conditions:
\begin{itemize}
\item If $T \in \mathcal{T}$ then $T' \in \mathcal{T}$ for any subdivision $T' \lhd T$.
\item If $T,T' \in \mathcal{T}$ then there exists $T'' \in \mathcal{T}$ such that both $T'' \lhd T$ and $T'' \lhd T'$.
\end{itemize}
We will simply write $X$ for a PL space $(X,\mathcal{T})$ if there is no danger of confusion.

\end{definition}


\begin{thebibliography}[
\bibitem{pltopo}Colin Patrick Rourke, Brian Joseph Sanderson, \textit{Introduction to piecewise-linear topology.}
\end{thebibliography}

\end{document}
